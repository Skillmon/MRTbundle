\chapter{The \cls{MRTbeam} class}
The \cls{MRTbeam} class is a class build upon \cls{beamer}. It should mimic the
style of the MS Powerpoint template of the MRT which was in use when I held my
Bachelor's presentation. I heard the requirements to match a specific
template are less strict today, but at least I'll still use this template.

Many of the features described here are also available if one uses
\cs{usetheme{MRTbeam}} within a document using the \cls{beamer} class. There
however is no dedicated documentation for that possibility provided. You're
encouraged to also use the corresponding \cls{MRTbeam} class if your using the
eponymous theme.

If there is a new institution template which should be matched that doesn't
match this \cls{beamer} template please contact me as mentioned in
\autoref{sec:bugs}.

\section{Options}
The class passes all options given to it on to \cls{beamer}. There are still
some options which you can set with some macros. The macros in this section are
only provided to set specific options, other macros are described in
\autoref{sec:beam:macros}.

\section{Macros}\label{sec:beam:macros}
\begin{describemacro}{PlaceAt}%
  [\meta{*}\barg{pos}\oarg{node options}\marg{content}]
  The starred version differs fundamentally from the unstarred one. The
  unstarred one places \meta{content} at the specified position \meta{pos} in
  the background inside a \TikZ\ node with the optionally specified \meta{node
  options}. The coordinates default to multiples of \cs{pagewidth} and
  \cs{pageheight} for \texttt{x} and \texttt{y}, respectively. You can use
  anything \TikZ\ understands as coordinates for \meta{pos}.

  The starred version places the \env{tikzpicture} where you currently are. It
  uses \texttt{remember picture} and \texttt{overlay} as options. The \meta{pos}
  must match the pattern \carg{x}{y}. \meta{x} is in multiples of \cs{pagewidth}
  and \meta{y} in multiples of \cs{pageheight} and you can't change that. The
  node still gets \meta{node options}.

  In both cases \texttt{(0,0)} is the bottom left corner of the slide.
\end{describemacro}

\begin{describemacro}{AddToPlaced}[\marg{tikz code}]
  Adds the specified \meta{tikz code} to the background of the current slide.
  \texttt{(0,0)} is the bottom left corner of the slide. Coordinates are by
  default in multiples of \cs{pagewidth} and \cs{pageheight}. It uses the same
  \env{tikzpicture} like \cs{PlaceAt} and is stored in the same macro.
\end{describemacro}

\begin{describemacro}{UseAndIfEmptyTF}%
  [\oarg{pre}\marg{arg}\marg{true}\marg{false}]
  The \meta{arg} is expanded inside a box. If that box has a width not equal 0pt
  \meta{pre} is used followed by the contents of the box. Then the \meta{false}
  branch is executed. If the box's width equals 0pt the \meta{true} branch is
  used instead and neither \meta{pre} is used nor the box containing \meta{arg}
  placed.
\end{describemacro}

\begin{describemacro}{cursec}[\meta{*}]
  If the current section is starred or you used the optional \texttt{*} for
  \cs{cursec}, this macro inserts the current sections name, else the name is
  prepended by the current sections number.
\end{describemacro}

\begin{describemacro}{curssec}[\meta{*}]
  This macro is very similar to \cs{cursec}. If you used the starred version of
  it or the current subsection is starred, this macro inserts the current
  subsections name, else the name is prepended by the current subsections
  number.
\end{describemacro}

\begin{describeenv}{whiteframes}
  In this environment \cs{ifwhiteframes} is set true.
\end{describeenv}

\subsection{Footnote related}
\begin{describemacro}{AddToRightFoot}%
  [\meta{*}\meta{+}\sarg{overlay}\oarg{pre}\marg{note}]
  This macro adds stuff to the right footer. If \meta{*} is given, the content
  is added to the persistent footnotes, else if \meta{+} is given added to the
  cite related footnotes, else to the ordinary ones. \meta{overlay} is used for
  any overlay specifications using \cs{uncover}. \meta{pre} is added left to
  \meta{note}. If tabular footnotes are used \meta{pre} is in the left,
  \meta{note} in the right column. If tabular footnotes are not used the
  distance between \meta{pre} and \meta{note} is \texttt{0.5}\cs{tabcolsep}. The
  starred variant should only be used outside of the \env{frame} environment. If
  you get strange errors during compilation a \cs{noexpand} in front of some
  macros (e.g. stuff like \cs{href}) you give as arguments might help.
\end{describemacro}

\begin{describemacro}{ClearRightFoot}[\meta{*}]
  Clears the footnotes. If the \texttt{*} is given only the volatile footnotes
  are cleared, else all of them.
\end{describemacro}

\subsection{Bibliography related}
\begin{describemacro}{cite}[\sarg{overlay}\oarg{opt1}\oarg{opt2}\meta{key}]
  \meta{overlay} is handled by \cs{uncover}, which affects only the footnote not
  the footnote mark. The usage of the two optional arguments and \meta{key}
  match those known from \pkg{biblatex}'s \cs{cite}.
\end{describemacro}

\begin{describemacro}{framecite}%
  [\meta{*}\sarg{overlay}\oarg{pre}\marg{key}\oarg{after}]
  Places a citation in 
\end{describemacro}
