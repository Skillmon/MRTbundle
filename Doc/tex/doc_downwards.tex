\chapter{Downwards Incompatibilities}
Though I try to avoid it as much as possible, there might sometimes be a
backwards incompatibility, which is necessary to implement new features I deem
worth it or to fix bugs. Where possible I'll try to provide means to switch back
to something close to the old behaviour.

This chapter should list all those ground breaking changes (which might be
really minor, though).

\begin{incompatibilities}{\cls{MRTthesis}}
  2019-11-20 & v0.0.18 & \cs{affidavit}
    & From this version on the affidavit is available using singular and plural
      forms, with the default being to automatically decide which one to use
      based on the given \opt{author}. You can deactivate this using
      \opt{affidavit plural=false} in \cs{MRTthesisSetup}. (See
      \autoref{sec:thesis:setup})
\end{incompatibilities}<++>

\begin{incompatibilities}{\pkg{MRTtab}}
  2019-02-09 & v0.0.5 & \cs{MRTcline}
    & From this version on by default the entire line is coloured first and then
    the effects of \cs{MRTcline} are applied easing the process of drawing
    interrupted lines. Reversible with the \opt{cline version} key (see
    \autoref{sec:tab:options:setup}). \\
\end{incompatibilities}

\begin{incompatibilities}{\pkg{MRTfonts}}
  2019-05-02 & v0.0.3 & \cs{altlUnscaled}
    & The entire macro got removed and will not come back. \\
  2019-05-02 & v0.0.3 & \cs{scalemath}
    & The macro is only available if the \opt{scale macro} option got used (see
    \autoref{sec:fonts:options}) \\
  2019-05-02 & v0.0.3 & \opt{new maths}
    & The \pkg{bm} package is no longer automatically loaded if you use the
    \opt{new maths} option. Additionally \pkg{MRTlmscale} and the maths fonts of
    \pkg{lmodern} will not be loaded at all. \\
  2019-05-02 & v0.0.3 & maths letter |l|
    & By default the \opt{alt l} option will be used replacing the letter |l| in
    maths by a letter distinct from an upper case |I|. You can revert to the
    previous behaviour using the \opt{std l} option (see
    \autoref{sec:fonts:options}) \\
\end{incompatibilities}
