\chapter{The \cls{MRTthesis} class}
\cls{MRTthesis} provides the template to write a thesis at the MRT. It sports a
layout which looks confusingly similar to the MS Word template provided by the
chair. Of course there are some minor differences and the typesetting algorithm
of \TeX\ should create better line breaking than Word's but if one doesn't know
what to pay attention on or for an untrained eye the distinction won't be
possible (at least I hope so, as that was the goal in the first place).

\section{Options}%>>=
\subsection{Load time options}\label{sec:thesis:opt}%>>=
The class features a few load time options.
\begin{describeopt}{longtable}
  Is forwarded to \pkg{MRTtab} see its description in \autoref{sec:tab:options}.
\end{describeopt}
\begin{describeopt}{hidelinks}
  If used the \pkg{hyperref} option of the same name will be used. By default
  this is used. You can negate it with \opt{showlinks}.
\end{describeopt}
\begin{describeopt}{minimal}
  If this option is passed some packages are not loaded and therefore related
  configurations not set. See \autoref{sec:thesis:dep}.
\end{describeopt}
\begin{describeopt}{no geometry}
  If this option is passed the \pkg{geometry} package is not loaded (and of
  course the page dimensions passed to \pkg{geometry} otherwise are not set).
\end{describeopt}
\begin{describeopt}{showlinks}
  If used the \pkg{hyperref} option \verb|hidelinks| will not be used. This is
  the negation of \opt{hidelinks} of this package.
\end{describeopt}
\begin{describeopt}{tikzunderline, tUline}
  This option is forwarded to \pkg{MRTwuline}.
\end{describeopt}
\begin{describeopt}{british,english,UKenglish}
  If used the document will be using the \opt{british} definition of
  \pkg{babel}. Many strings used in the package will be in English, but some
  might be missed out. If you find any of which you think should be translated,
  please contact me as described in \autoref{sec:bugs}. English simplified (US)
  is not supported by the class.
\end{describeopt}
The following options (and their values) will be forwarded to \pkg{MRTfonts}
(see \autoref{sec:fonts:options} for their description):
\begingroup
\multicolsep=0pt
\premulticols=0pt
\begin{multicols}{4}
\begin{itemize}
  \item \opt{sfacc}
  \item \opt{font}
  \item \opt{serif font}
  \item \opt{mono font}
  \item \opt{new maths}
  \item \opt{scale maths}
  \item \opt{scale macro}
  \item \opt{alt l}
  \item \opt{std l}
  \item \opt{mathsizes}
  \item \opt{no mathsizes}
  \item \opt{pmb}
\end{itemize}
\end{multicols}
\endgroup


\noindent
Every other given option will be passed on to \cls{scrreprt}.
%=<<
\subsection{Setup options}\label{sec:thesis:setup}%>>=
The following options are accessible with \cs{MRTthesisSetup}.
\begin{describeopt}{advisor}[\meta{name}]
  Sets the name of the advisor of this thesis. One typical value could be
  \texttt{Dipl.-Ing.\@ Alice Fischerauer}
\end{describeopt}
\begin{describeopt}{author}[\meta{name}]
  Sets the name of the author or authors as the macro \cs{author} does.
  Separate authors with \cs{and}. You can give the surname first followed by a
  comma and the given name, in which case the parsing for the abbreviation works
  better (especially with name affixes). The following two options are fine:
  \texttt{author=\{Duck, Donald \string\and\ Mouse, Mickey\}} or
  \texttt{author=\{Donald Duck \string\and\ Mickey Mouse\}}; both should result
  in the abbreviation \texttt{D.\@ Duck, M.\@ Mouse}. Another example would be
  \texttt{zu Guttenberg, Karl-Theodor} or \texttt{Karl-Theodor zu Guttenberg}.
  Here the parsing would result in \texttt{K.-T.\@ zu Guttenberg} or
  \texttt{K.-T.\@ z.\@ Guttenberg} -- the first one seems correct, the second
  one fails. Remember to surround the argument with braces if you use a comma.
\end{describeopt}
\begin{describeopt}{caption above}
  Is forwarded to \pkg{MRTtab}. See \autoref{sec:tab:options:setup}.
\end{describeopt}
\begin{describeopt}{caption below}
  Is forwarded to \pkg{MRTtab}. See \autoref{sec:tab:options:setup}.
\end{describeopt}
\begin{describeopt}{citation width}[\meta{dimen}]
  The width of the citation indications on the title page. Default is
  \verb|.5\textwidth|.
\end{describeopt}
\begin{describeopt}{degree}[\meta{degree}]
  The degree you aim to achieve with the thesis. If you don't use this option it
  is tried to be guessed from the type of thesis you can specify with the
  \opt{thesis} key. An error is thrown if the degree can't be guessed. If you
  don't want to achieve any degree, use the option \opt{no degree}. Typical
  values would be \verb|Bachelor of Science| or \verb|Master of Science|.
\end{describeopt}
\begin{describeopt}{examiner}[\meta{name}]
  The examiner of the thesis. The initial value is set to
  \texttt{Univ.-Professor Dr.-Ing.\@ Gerhard Fischerauer}.
\end{describeopt}
\begin{describeopt}{logoL}[\meta{file}]
  The image file for the left logo on the titlepage.
  \file{MRTresources_logo_UBT2.pdf} is the initial value. If \meta{file} is an
  empty argument no left logo will be used.
\end{describeopt}
\begin{describeopt}{logoL height}[\meta{dimen}]
  The height the left logo is displayed in. Initial value is \verb|10.85mm|.
\end{describeopt}
\begin{describeopt}{logoR}[\meta{file}]
  The image file for the right logo on the titlepage.
  \file{MRTresources_logo_MRT2.}\-\file{pdf} is the initial value. If
  \meta{file} is an empty argument no right logo will be used.
\end{describeopt}
\begin{describeopt}{logoR height}[\meta{dimen}]
  The height the right logo is displayed in. Initial value is \verb|11.9mm|.
\end{describeopt}
\begin{describeopt}{no advisor}[\meta{bool}]
  If true no advisor will be displayed on the title page. Default is true,
  initially is false.
\end{describeopt}
\begin{describeopt}{no citation}[\meta{bool}]
  If true no citation indications are displayed at the bottom of the title page.
  Default is true, initially is false.
\end{describeopt}
\begin{describeopt}{no degree}[\meta{bool}]
  If true no degree will be displayed on the title page. Default is true,
  initially is false. Also the paragraph corresponding to the degree in the
  affidavit will be left out.
\end{describeopt}
\begin{describeopt}{no examiner}[\meta{bool}]
  If true no examiner will be displayed on the title page. Default is true,
  initially is false.
\end{describeopt}
\begin{describeopt}{no chair}[\meta{bool}]
  If true no chair will be displayed on the title page. Default is true,
  initially is false.
\end{describeopt}
\begin{describeopt}{no logos}
  If used \verb|logoL={},logoR={}| is used, which results in no logos on the
  title page.
\end{describeopt}
\begin{describeopt}{no thesis}[\meta{bool}]
  If true no thesis type will be displayed on the title page. Default is true,
  initially is false.
\end{describeopt}
\begin{describeopt}{no usage}[\meta{bool}]
  If true no usage rights are given to the MRT in the affidavit text. Default is
  true, initially is false. If you need a custom paragraph and don't want to
  leave it out completely you should redefine \cs{affidavittext@usagerights}.
\end{describeopt}
\begin{describeopt}{number}[\meta{number}]
  The MRT report number displayed in the citation indications. Initially is
  empty. The typical pattern of these numbers is something like:
  \texttt{TT-yy-mm-nn} with \texttt{TT} the thesis type, e.\,g.\@ \texttt{BA} or
  \texttt{MA}, \texttt{yy} the last two digits of the year, \texttt{mm} the
  month, and \texttt{nn} the number of the thesis in this month.
\end{describeopt}
\begin{describeopt}{pos figure}[\meta{placement}]
  The \meta{placement} of floats of type \texttt{figure}.
\end{describeopt}
\begin{describeopt}{pos float}[\meta{placement}]
  The \meta{placement} of floats of both types, \texttt{figure} and
  \texttt{table}. Initially set to \texttt{tbp}.
\end{describeopt}
\begin{describeopt}{pos MRTtable}[\meta{placement}]
  The \meta{placement} of floating \env{MRTtable}s, forwarded to \pkg{MRTtab}'s
  option \opt{pos}. See \autoref{sec:tab:options:setup}.
\end{describeopt}
\begin{describeopt}{pos table}[\meta{placement}]
  The \meta{placement} of floats of type \texttt{table}.
\end{describeopt}
\begin{describeopt}{short advisor, sadvisor}[\meta{abbreviation}]
  The abbreviated name of the advisor. This is needed for the citation
  indications and not parsed automatically from the \opt{advisor}, as the name
  contains academic titles, but the abbreviation should not and the parsing
  would be hard to do correctly.
\end{describeopt}
\begin{describeopt}{short author, sauthor}[\meta{abbreviation}]
  The abbreviated name or names of the author or authors. If you don't use this
  option it is tried to parse those automatically. If the parsing does something
  wrong you'll have to use this option giving the correct abbreviations with
  each name separated with commas from the others, e.\,g.\@ 
  \texttt{short author=\{D. Duck, M. Mouse\}}.
\end{describeopt}
\begin{describeopt}{short examiner, sexaminer}[\meta{abbreviation}]
  The abbreviated name of the examiner. This is needed for the citation
  indications and not parsed automatically from the \opt{examiner}, as the name
  contains academic titles, but the abbreviation should not and parsing would be
  hard to do correctly. Initial value is \texttt{G.\@ Fischerauer}.
\end{describeopt}
\begin{describeopt}{sign height}[\meta{dimen}]
  The height reserved for each signature below the affidavit text. Initial value
  is \texttt{9mm}.
\end{describeopt}
\begin{describeopt}{sign separation, sign sep}[\meta{dimen}]
  If \opt{sign width max} is not given (or \texttt{0pt}) the maximum width is
  calculated from the text width and the width of the date and location. The
  minimum distance from the date and location to the signature lines is then
  enforced to be at least \meta{dimen}. Initial value is \texttt{2em}.
\end{describeopt}
\begin{describeopt}{sign width max}[\meta{dimen}]
  You can enforce a maximum width for the signature lines below the affidavit
  using this option. If it is not used, the maximum width is calculated.
\end{describeopt}
\begin{describeopt}{sign width min}[\meta{dimen}]
  You can enforce a minimum width for the signature lines using this option.
  Initially this is set to \texttt{7cm}.
\end{describeopt}
\begin{describeopt}{stretch caption, stretch cap}[\meta{float}]
  Uses \cs{setkomafont} to enforce a specific line spread using \cs{setstretch}
  for captions.
\end{describeopt}
\begin{describeopt}{stretch tabular, stretch tab}[\meta{float}]
  Is forwarded to \pkg{MRTtab}'s option \opt{stretch tab}. See
  \autoref{sec:tab:options:setup}.
\end{describeopt}
\begin{describeopt}{stretch text}[\meta{float}]
  Uses \cs{setstretch} to set a specific line spread in the document.
\end{describeopt}
\begin{describeopt}{stretches}[\meta{float}]
  Sets \opt{stretch cap}, \opt{stretch tab}, and \opt{stretch text} in one go.
  Initially set to \texttt{1.408}.
\end{describeopt}
\begin{describeopt}{subtitle}[\meta{title}]
  The title page might include a subtitle. If you really want to use it, you'd
  have to use \opt{with subtitle}. You can also use \cs{subtitle} to set it.
\end{describeopt}
\begin{describeopt}{thesis}[\meta{thesis type}]
  Sets the \meta{thesis type}. Typical arguments would be
  \texttt{Bachelorarbeit} or \texttt{Bachelor Thesis} (the former if you're
  writing in German, the latter if you're writing in English).
\end{describeopt}
\begin{describeopt}{title}[\meta{title}]
  Sets the title of the thesis. You might also use \cs{title} to set this.
\end{describeopt}
\begin{describeopt}{toc ChapIndent}[\meta{dimen}]
  Sets the indentation of chapter entries in the table of contents. Initially
  set to \texttt{0.01em}.
\end{describeopt}
\begin{describeopt}{toc SecIndent}[\meta{dimen}]
  Sets the indentation of section entries in the table of contents. Initially
  set to \texttt{1.32em}. The width is also used for entries in the list of
  figures and list of tables.
\end{describeopt}
\begin{describeopt}{toc sSecIndent}[\meta{dimen}]
  Sets the indentation of subsection entries in the table of contents. Initially
  set to \texttt{3.38em}.
\end{describeopt}
\begin{describeopt}{toc ssSecIndent}[\meta{dimen}]
  Sets the indentation of subsubsection entries in the table of contents.
  Initially set to \texttt{6.38em}.
\end{describeopt}
\begin{describeopt}{with subtitle}[\meta{bool}]
  If true a subtitle can be used on the title page. Default is true, initially
  is false.
\end{describeopt}

\subsubsection{Options concerning automatically added contents}%>>=
\label{sec:autoopt}
The following additional options can be set with \cs{MRTthesisSetup}. They all
resolve around automatically added contents.
\begin{describeopt}{backmatter}[\meta{choice}]
  A \meta{choice} whether you want the back matter to be added automatically.
  Possible values are |auto| and |manual|. If set to |auto| the appendix will
  automatically be included at |\end{document}|. It might contain the following
  (dependent on the values of other keys; in correct order):
  \begin{fakeitemize}
    \item bibliography (option |bib|)
    \item list of figures (option |lof|)
    \item list of tables (option |lot|)
    \item contents added with \cs{MRTthesisAddToBack}
    \item the contents of your appendix file (option |appendix|)
    \item contents added with \cs{MRTthesisAddAfterBack}
    \item the affidavit (option |affidavit|)
  \end{fakeitemize}
  It also includes the necessary formatting switches otherwise contained in
  \cs{appendix}. Default is |manual|.
\end{describeopt}
\begin{describeopt}{frontmatter}[\meta{choice}]
  A \meta{choice} whether you want the front matter to be added automatically.
  Possible values are |auto| and |manual|. If set to |auto| the front matter
  will automatically be included at |\begin{document}|. It might contain the
  following (dependent on the values of other keys; in correct order):
  \begin{fakeitemize}
    \item title page
    \item the affidavit (option |affidavit|)
    \item the acknowledgements (option |acknowledgement|)
    \item table of contents (option |toc|)
    \item list of figures (option |lof|)
    \item list of tables (option |lot|)
    \item contents added with \cs{MRTthesisAddToFront}
  \end{fakeitemize}
  It also includes the necessary formatting switches otherwise contained in
  \cs{mainpart}. Default is |manual|.
\end{describeopt}
\begin{describeopt}{acknowledgement}[\meta{file}]
  Sets the acknowledgements file added if \bverb|frontmatter=auto| is used. If
  \meta{file} (the argument) is empty no file will be added. By default it is
  empty. The \meta{file} will be added using \cs{include}, so you should drop
  the \bverb|.tex| ending.
\end{describeopt}
\begin{describeopt}{affidavit}[\meta{choice}]
  Sets where the \cs{affidavit} is added. Possible \meta{choice}s are |front|,
  |back| and |off|. If |off| is used it doesn't get added automatically. Default
  value is |front|. |front| and |back| will only take effect if |frontmatter|
  and |backmatter| are set to |auto|, respectively.
\end{describeopt}
\begin{describeopt}{appendix}[\meta{file}]
  Sets the appendix file added if \bverb|backmatter=auto| is used. If
  \meta{file} (the argument) is empty no file will be added. By default it is
  empty. The \meta{file} will be added using \cs{include}, so you should drop
  the \bverb|.tex| ending.
\end{describeopt}
\begin{describeopt}{appendix ragged}[\meta{bool}]
  If set |true| the contents of the |appendix| file will be typeset
  \cs{raggedbottom}. Default is |true|.
\end{describeopt}
\begin{describeopt}{bib,bibliography}[\meta{bool}]
  Sets whether the bibliography should be added automatically if
  \bverb|backmatter=auto| is used. It gets set to |false| if the
  class option |minimal| is used. 
\end{describeopt}
\begin{describeopt}{lof}[\meta{choice}]
  Sets where the list of figures is added. Possible \meta{choice}s are |front|,
  |back| and |off|. If |off| is used it doesn't get added automatically. Default
  value is |front|. |front| and |back| will only take effect if |frontmatter|
  and |backmatter| are set to |auto|, respectively.
\end{describeopt}
\begin{describeopt}{lot}[\meta{choice}]
  Sets where the list of tables is added. Possible \meta{choice}s are |front|,
  |back| and |off|. If |off| is used it doesn't get added automatically. Default
  value is |front|. |front| and |back| will only take effect if |frontmatter|
  and |backmatter| are set to |auto|, respectively.
\end{describeopt}
\begin{describeopt}{toc}[\meta{choice}]
  Sets where the list of tables is added if |frontmatter=auto| is used. Possible
  \meta{choice}s are |front| and |off|. If |off| is used it doesn't get added
  automatically. Default value is |front|.
\end{describeopt}
%=<<
%=<<
%=<<

\section{Macros}\label{sec:thesis:mac}%>>=
The following macros are provided:
\begin{describemacroTF}{ifNoWidth}[\marg{arg}]
  Typesets the argument in a box (so the code is actually executed). If the
  produced box has a width of 0pt the \meta{true} branch is executed, else the
  \meta{false} branch.
\end{describemacroTF}
\begin{describemacro}{vfillmult}[\marg{num}]
  Same as if you'd use \meta{num} instances of \cs{vfill}.
\end{describemacro}
\begin{describemacro}{hfillmult}[\marg{num}]
  Same as if you'd use \meta{num} instances of \cs{hfill}.
\end{describemacro}
\begin{describemacro}{MRTafterhyperref}[\marg{content}]
  Places \meta{content} after \pkg{hyperref} is loaded. This is important for
  the relatively few packages that need to be loaded after \pkg{hyperref}. So if
  you have one of these, you should use something like
  \bverb|\MRTafterhyperref{\usepackage{cleveref}}|. This macro has to be used
  prior to \bverb|\begin{document}|.
\end{describemacro}
\begin{describemacro}{MRTthesisAddToFront}[\marg{content}]
  Adds \meta{content} to a hook executed during the front matter if
  \bverb|frontmatter=auto| was used. See \autoref{sec:autoopt} for more
  information.
\end{describemacro}
\begin{describemacro}{MRTthesisAddToBack}[\marg{content}]
  Adds \meta{content} to a hook executed during the back matter if
  \bverb|backmatter=auto| was used. See \autoref{sec:autoopt} for more
  information.
\end{describemacro}
\begin{describemacro}{MRTthesisAddAfterBack}[\marg{content}]
  Adds \meta{content} to a hook executed during the back matter if
  \bverb|backmatter=auto| was used. See \autoref{sec:autoopt} for more
  information.
\end{describemacro}
\begin{describemacro}{MRTthesisAddToAppendixSwitch}[\marg{content}]
  Adds \meta{content} to the \cs{appendix} switch. This is useful if you need
  the behaviour of some other packages changed during the appendix.
\end{describemacro}
\begin{describemacro}{DeclareTOCStyleEntryMRTChapterLike}%
  [\oarg{indent}\hskip0pt\marg{entry-layer}\hskip0pt\oarg{options}]
  See the description of \cs{DeclareTOCStyleEntryMRTSectionLike}. The difference
  is that this sets the entries how the chapters are formatted. Also the
  \meta{indent} defaults to the one of chapters.
\end{describemacro}
\begin{describemacro}{DeclareTOCStyleEntryMRTSectionLike}%
  [\oarg{indent}\hskip0pt\marg{entry-layer}\hskip0pt\oarg{options}]
  The macro calls the KOMA macro \cs{DeclareTOCStyleEntry} and sets the options
  how they are used for the section entries in the table of contents.
  \meta{indent} defaults to the indent length of section entries. It is possible
  to use an \meta{indent} but give more options in that optional argument
  afterwards (comma separated). \autoref{tab:thesis:tocindents} shows an
  overview of the class's default indentations. With the \meta{options} argument
  you can specify further options past the indent to \cs{DeclareTOCStyleEntry}.
\end{describemacro}
\begin{MRTtable}%>>=
  [%>>=
    ,label=tab:thesis:tocindents
    ,cap=
      {
        Indents of different ToC entry types and the macros they are stored in.
        Use the options described in \autoref{sec:thesis:setup} to change the
        values.
      }
    ,scap=
      {Indents of different ToC entry types and the macro they are stored in}
    ,col=llc
  ]%=<<
  entry-layer & macro name & default length \\

  chapter       & \cs{l_MRTthesis_toc_chapter_indent_tl}       & 0.01em\\
  section       & \cs{l_MRTthesis_toc_section_indent_tl}       & 1.32em\\
  subsection    & \cs{l_MRTthesis_toc_subsection_indent_tl}    & 3.38em\\
  subsubsection & \cs{l_MRTthesis_toc_subsubsection_indent_tl} & 6.38em\\
  table         & \cs{l_MRTthesis_toc_section_indent_tl}       & 1.32em\\
  figure        & \cs{l_MRTthesis_toc_section_indent_tl}       & 1.32em\\
\end{MRTtable}%=<<
\begin{describemacro}{MRTthesisSetup}[\marg{options}]
  You can use this macro to set the options listed in
  \autoref{sec:thesis:setup}.
\end{describemacro}
\begin{describemacro}{sethead}[\marg{content}]
  Sets the head marks for both sides to \meta{content}. It is the same as
  \bverb|\markboth{|\meta{content}\bverb|}{|\meta{content}\bverb|}|. You might
  use this (or any similar macro provided by KOMA script) to manually set the
  head marks, e.\,g.\@ if your section title gets too long.
\end{describemacro}
\begin{describemacro}{affidavit}
  Prints a chapter ``Eidesstattliche Erklärung'' (stored in \cs{affidavittitle},
  you might redefine it to change the title) and the affidavit text (as stored
  in \cs{affidavittext}) and the location and date, followed by a signature line
  for each author. \cs{maketitle} has to be used prior to it, else the lines
  won't be printed. This is a bug I might fix in the future.
\end{describemacro}
\begin{describemacro}{mainpart,mainmatter}
  Switches the formatting from the one at the beginning to the one used in the
  main part of the document. Should be used after \cs{tableofcontents},
  \cs{listoffigures}, and \cs{listoftables}.
\end{describemacro}
\begin{describemacro}{appendix}
  Switches the formatting to the one used in the appendix. This includes
  switching to alphabetically numbered sections and setting the option \opt{no
  float} in \cs{MRTtabSetup}.
\end{describemacro}
Additionally the macros \cs{author}, \cs{title}, and \cs{subtitle} have been
redefined to internally use \cs{MRTthesisSetup} to set the corresponding
options. \cs{author} takes an optional argument with which you can set the
abbreviated list of authors.
%=<<

\section{Dependencies}\label{sec:thesis:dep}%>>=
As this class is based on \cls{scrreprt}, it depends on that class and all of
its dependencies, of course. Additionally the following packages are loaded
(used options given in brackets). Those are quite some but unfortunately most of
these are required (or help a lot) to achieve certain formattings in order to
match the MS Word template of the MRT best.

Some of the used packages are not necessarily needed to match the MS Word
template, but provide useful features -- e.g.  \pkg{hyperref} which allows the
use of \cs{autoref} and cross linking but is not needed to match any specific
formatting.%
\footnote
  {Don't remove it though, the current code for section headings relies on it.}

\begin{multicols}{2}%>>=
  \begin{itemize}[leftmargin=10pt]
    \item \pkg{expl3}
    \item \pkg{xparse}
    \item \pkg{MRTif}
    \item \pkg{MRTutil}
    \item \pkg{MRTsfacc}
    \item \pkg{MRTtab}
    \item \pkg{MRTwuline}
    \item \pkg{MRTfonts}
    \item \pkg{babel} [ngerman] or if \opt{british} is used with
      [main\Seq british, ngerman]
    \item \pkg{scrlayer-scrpage} [singlespacing\Seq true]
    \item \pkg{geometry} (with correct options)
    \item \pkg{setspace}
    \item \pkg{xcolor}
    \item \pkg{graphicx}
    \item \pkg{enumitem}
    \item \pkg{mathtools}
    \item \pkg{mathastext} [italic,defaultmathsizes]
    \item \pkg{isomath}
    \item \pkg{hyperref}
    \item if the \opt{minimal} option is not used:
      \begin{itemize}
        \item \pkg{siunitx}
          [%
            detect-all,
            per-mode\Seq reciprocal-positive-first%
          ]\\
          If \pkg{babel}'s \opt{british} is used [locale\Seq UK] will be used,
          if \opt{ngerman} [locale\Seq DE]. Additionally the \opt{range-phrase}
          will be set to either \texttt{to} or \texttt{bis} with spaces around
          it.
        \item \pkg{biblatex}
          [%
            backend\Seq biber, natbib\Seq true, citestyle\Seq numeric,
            bibstyle\Seq numeric, sorting\Seq none, giveninits\Seq true,
            sortcites%
          ] (with URLs being line breakable at any place)
        \item \pkg{csquotes}
      \end{itemize}
  \end{itemize}
\end{multicols}%=<<
%=<<

% vim: ft=tex fdm=marker fmr=>>=,=<< sw=2 ts=2 tw=80
