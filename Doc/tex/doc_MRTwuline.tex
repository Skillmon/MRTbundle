\chapter{The \pkg{MRTwuline} package}
The package provides a MS Word like looking line breakable underlining. It does
so by using \pkg{ulem} or \pkg{stackengine}, and if \pkg{lua-ul} is available we
use that (so this package gives best results with \hologo{LuaLaTeX}), but see
the |luaul| option in \autoref{sec:wuline:options:setup}.

\section{Options}\label{sec:wuline:options}
\subsection{Load time options}\label{sec:wuline:options:loadtime}%>>=
\begin{describeopt}{tUline,tikzunderline}
  If this option is passed \TikZ\ will be added as a required package and an
  alternative underlining macro defined called \cs{tUline}, see its description
  in \autoref{sec:wuline:mac}.
\end{describeopt}
%=<<

\subsection{Setup options}\label{sec:wuline:options:setup}%>>=
\begin{describeopt}{rule~ht,ruleht}[\meta{dimen}]
  With this setting you can decide how thick the rule is, the initial value is
  \texttt{\csuse{l__MRTwuline_rule_thickness_tl}}, the value is evaluated at use
  time, so that a smaller font gets a thinner line.
\end{describeopt}

\begin{describeopt}{math~dp,mathdp}[\meta{dimen}]
  With this setting you can decide the height of the rule below the symbol in
  maths mode. The initial value is
  \texttt{\csuse{l__MRTwuline_math_depth_tl}}, the value is evaluated at use
  time.
\end{describeopt}

\begin{describeopt}{text~dp,textdp}[\meta{dimen}]
  With this setting you can decide the height of the rule below the symbol in
  text mode. The initial value is
  \texttt{\csname l__MRTwuline_text_depth_tl\endcsname}, the value is evaluated
  at use time.
\end{describeopt}

\begin{describeopt}{luaul,lua-ul}[\meta{bool}]
  If you're using \hologo{LuaLaTeX}, the \pkg{lua-ul} package is loaded. With it
  \cs{WUline} behaves much better, giving the typesetting results you'd get
  without it in text, just with underlining (this includes automatic
  hyphenation). With this key you can decide whether \cs{WUline} uses
  \pkg{lua-ul} or not, if available it is initially set to |true|.
\end{describeopt}
%=<<

\section{Macros}\label{sec:wuline:mac}%>>=
\begin{describemacro}{WUline}[\oarg{height}\marg{text}]%>>=
  This sets \meta{text} and underlines it in a way that looks like MS Word
  underlining -- at least in the headings. It is usable both in math mode and in
  text mode. Though in math mode you should probably use
  \cs{underline}.\\[\parskip]
  In text mode the \pkg{ulem} package (or \pkg{lua-ul}) is used for the
  underline. In math mode \pkg{stackengine} is employed. In both cases you can
  use \meta{height} to change the default height of the underlining. In text
  mode and math mode the needed \meta{height} to achieve the same height of the
  line differs quite a lot. By default in math mode the value given to the
  |mathdp| option is used, in text mode the value of |textdp| (see
  \autoref{sec:wuline:options:setup}).\\[\parskip]
  While the text mode version using \pkg{ulem} is line breakable it disables
  automatic hyphenation in its argument, you can still use \cs{-} to set
  hyphenation points manually.
\end{describemacro}%=<<
\begin{describemacro}{tUline}%>>=
  [\oarg{height}\oarg{overhang}\oarg{thickness}\marg{text}]
  This macro can be used to underline bigger portions of text. You should never
  need it, I guess. Just use \cs{WUline} instead. If you need it, you'll have to
  use the package option \opt{tUline} (see
  \autoref{sec:wuline:options:loadtime}).\\[\parskip]
  If you think you can use this one instead: It underlines \meta{text} at the
  given \meta{height} (default \texttt{-0.35ex}) with the given \meta{thickness}
  (default \texttt{0.185ex}). You can specify \meta{overhang} (default
  \texttt{0pt}) which is the width the line should be wider than a text line on
  each side. If you let any optional argument empty, the default is used. It is
  assumed that the lines are equally separated with \cs{baselineskip} -- so if
  your material does stretch the baseline skip, you can't use \cs{tUline}. It
  needs at least two runs to be displayed correctly.
\end{describemacro}%=<<
%=<<

\section{Dependencies}%>>=
\begin{multicols}{2}
  \begin{itemize}[leftmargin=10pt]
    \item \pkg{expl3}
    \item \pkg{xparse}
    \item \pkg{stackengine}
    \item \pkg{scalerel}
    \item \pkg{MRTif}
    \item \pkg{MRTutil}
    \item \pkg{ulem} with the \opt{normalem} option
    \item if the \opt{tUline} option is used:
      \begin{itemize}
        \item \TikZ
        \item \pkg{tikzpagenodes}
        \item The \TikZ\ library \pkg{calc}
      \end{itemize}
    \item \pkg{lua-ul} if \hologo{LuaLaTeX} is used
  \end{itemize}
\end{multicols}
%=<<

% vim: ft=tex fdm=marker fmr=>>=,=<< sw=2 ts=2 tw=80
