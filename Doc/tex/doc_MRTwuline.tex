\chapter{The \pkg{MRTwuline} package}
The package provides a MS Word like looking line breakable underlining. It does
so by using \pkg{ulem} or \pkg{stackengine}.

\section{Options}\label{sec:wuline:options}%>>>
\begin{describeopt}{tUline,tikzunderline}
  If this option is passed \TikZ\ will be added as a required package and an
  alternative underlining macro defined called \cs{tUline}, see its description
  in \autoref{sec:wuline:mac}.
\end{describeopt}
%<<<

\section{Macros}\label{sec:wuline:mac}%>>>
\begin{describemacro}{WUline}[\oarg{height}\marg{text}]%>>>
  This sets \meta{text} and underlines it in a way that looks like MS Word
  underlining -- at least in the headings. It is usable both in math mode and in
  text mode. Though in math mode you should use \cs{underline}.\\[\parskip]
  In text mode the \pkg{ulem} package is used for the underline. In math mode
  \pkg{stackengine} is employed. In both cases you can use \meta{height} to
  change the default height of the underlining. In text mode and math mode the
  needed \meta{height} to achieve the same height of the line differs quite a
  lot. By default in math mode \texttt{\csuse{MRTwuline@mathheight}} is
  used, in text mode \texttt{\csuse{MRTwuline@textheight}}.
\end{describemacro}%<<<
\begin{describemacro}{tUline}%>>>
  [\oarg{height}\oarg{overhang}\oarg{thickness}\marg{text}]
  This macro can be used to underline bigger portions of text. You should never
  need it, I guess. Just use \cs{WUline} instead. If you need it, you'll have to
  use the package option \opt{tUline}.\\[\parskip]
  If you think you can use this one instead: It underlines \meta{text} at the
  given \meta{height} (default \texttt{-0.35ex}) with the given \meta{thickness}
  (default \texttt{0.185ex}). You can specify \meta{overhang} (default
  \texttt{0pt}) which is the width the line should be wider than a text line on
  each side. If you let any optional argument empty, the default is used. It is
  assumed that the lines are equally separated with \cs{baselineskip} -- so if
  your material does stretch the baseline skip, you can't use \cs{tUline}. It
  needs at least two runs to be displayed correctly.
\end{describemacro}%<<<
%<<<

\section{Dependencies}%>>>
\begin{multicols}{2}%>>>
  \begin{itemize}[leftmargin=10pt]
    \item \pkg{expl3}
    \item \pkg{xparse}
    \item \pkg{stackengine}
    \item \pkg{scalerel}
    \item \pkg{MRTif}
    \item \pkg{MRTutil}
    \item \pkg{ulem} with the \opt{normalem} option
    \item if the \opt{tUline} option is used:
      \begin{itemize}
        \item \TikZ
        \item \pkg{tikzpagenodes}
        \item The \TikZ\ library \pkg{calc}
      \end{itemize}
  \end{itemize}
\end{multicols}
%<<<
