\chapter{The \pkg{MRTutil} package}\label{sec:util}
This package provides some utility functions. Those are meant to aid people
defining their own macros and are used throughout other packages of this bundle.
Every macro this package provides is at the code level so there are no real user
facing macros. As a general rule of thumb the user level is therefore moved one
layer down, user facing macros have a single \texttt{@} in their names, while
internal macros have at least two.

\section{Defining Macros}
Since the author of this bundle often finds the possibilities of the \LaTeX2e
macro family of \cs{newcommand} too restricting, the package provides some
macros which use the syntax of \TeX's \cs{def} but still check whether the macro
is already defined.

\begin{describemacro}{MRTutil@def,MRTutil@edef}%
  [\oarg{prefixes}\meta{cs}\meta{args}\marg{definition}]
  Those are versions of \cs{def} and \cs{edef}. You can define \meta{prefixes}
  like \cs{long} or \cs{protected}. \meta{cs} is the new control sequence's
  name, \meta{args} is the argument specification and \meta{definition} is the
  replacement text of the macro. Both check whether the macro is already defined
  and will raise an error if they are. They are like a fusion of \cs{newcommand}
  and \cs{def} in that they only define a new command but keep the versatility
  of \cs{def}.
\end{describemacro}

\section{Optional Argument Parsing}
Since the author really likes what \pkg{xparse} allows in defining macros with
many optional arguments but doesn't want to force the complete \pkg{expl3} onto
the user (since it's huge), if a user is only interested in one or two of the
small packages of this bundle there are some macros in those packages which have
multiple optional arguments. To define those the following macros were created
to provide a very limited subset of \pkg{xparse}'s functionality. Note that none
of the provided macros allow expandable definitions.

\begin{describemacro}{MRTutil@Oarg}[\marg{default}\marg{continue}]
  checks for a following optional argument in \texttt{[]}. If there is none it
  provides the \meta{default}. \meta{continue} will be executed after the
  argument has been parsed. The value of the optional argument will be provided
  to \meta{continue} in braces (\texttt{\{\}}).
\end{describemacro}

\begin{describemacro}{MRTutil@oarg}[\marg{continue}]
  Like \cs{MRTutil@Oarg}, but provides a special marker if there is no optional
  argument. You can check whether the special marker was provided with
  \cs{MRTutil@ifmark}.
\end{describemacro}

\begin{describemacro}{MRTutil@Darg}%
  [\meta{token1}\meta{token2}\marg{default}\marg{continue}]
  Like \cs{MRTutil@Oarg}, but the optional argument should be delimited by
  \meta{token1} and \meta{token2}. So \bverb|\MRTutil@Darg<>{}\foo| will check
  whether there is an optional argument delimited by \texttt{<>} and if there is
  none will use an empty one. The result is provided to \cs{foo}.
\end{describemacro}

\begin{describemacro}{MRTutil@darg}%
  [\meta{token1}\meta{token2}\marg{continue}]
  Like \cs{MRTutil@darg}, but provides a special marker if there is no optional
  argument. You can check whether the special marker was provided with
  \cs{MRTutil@ifmark}. 
\end{describemacro}

\begin{describemacro}{MRTutil@ifmark}[\marg{test}\marg{true}\marg{false}]
  Tests whether \marg{test} is the special mark provided by \cs{MRTutil@oarg}
  and \cs{MRTutil@darg} and if so expands to \meta{true}. If not it expands to
  \meta{false}.
\end{describemacro}
