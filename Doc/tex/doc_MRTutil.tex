\chapter{The \pkg{MRTutil} package}\label{sec:util}
This package provides some utility functions. Those are meant to aid people
defining their own macros and are used throughout other packages of this bundle.
Every macro this package provides is at the code level so there are no real user
facing macros. As a general rule of thumb the user level is therefore moved one
layer down, user facing macros have a single \texttt{@} in their names, while
internal macros have at least two.

\section{Defining Macros}%>>=
Since the author of this bundle often finds the possibilities of the \LaTeXe\
macro family of \cs{newcommand} too restricting, the package provides some
macros which use the syntax of \TeX's \cs{def} but still check whether the macro
is already defined.

\begin{describemacro}{MRTutil@def,MRTutil@edef}%
  [\oarg{prefixes}\meta{cs}\meta{args}\marg{definition}]
  Those are versions of \cs{def} and \cs{edef}. You can define \meta{prefixes}
  like \cs{long} or \cs{protected}. \meta{cs} is the new control sequence's
  name, \meta{args} is the argument specification and \meta{definition} is the
  replacement text of the macro. Both check whether the macro is already defined
  and will raise an error if they are. They are like a fusion of \cs{newcommand}
  and \cs{def} in that they only define a new command but keep the versatility
  of \cs{def}. There is a usage example in \autoref{sec:util:example}.
\end{describemacro}
%=<<

\section{Optional Argument Parsing}%>>=
Since the author really likes what \pkg{xparse} allows in defining macros with
many optional arguments but doesn't want to force the complete \pkg{expl3} onto
the user (since it's huge), if a user is only interested in one or two of the
small packages of this bundle there are some macros in those packages which have
multiple optional arguments. To define those the following macros were created
to provide a very limited subset of \pkg{xparse}'s functionality. Note that none
of the provided macros allow expandable definitions and none checks for matched
delimiters (so if you want to pass in a |[| and a |]| to a macro with an
optional argument delimited by |[]| you'll need to use |[{[foo]}]|, this is the
same behaviour encountered in \LaTeXe's optional arguments).

\begin{describemacro}{MRTutil@Oarg}[\marg{default}\marg{continue}]
  checks for a following optional argument in \texttt{[]}. If there is none it
  provides the \meta{default}. \meta{continue} will be executed after the
  argument has been parsed. The value of the optional argument will be provided
  to \meta{continue} in braces (\texttt{\{\}}).
\end{describemacro}

\begin{describemacro}{MRTutil@oarg}[\marg{continue}]
  Like \cs{MRTutil@Oarg}, but provides a special marker if there is no optional
  argument. You can check whether the special marker was provided with
  \cs{MRTutil@ifmark}. There is a usage example in \autoref{sec:util:example}.
\end{describemacro}

\begin{describemacro}{MRTutil@Darg}%
  [\meta{token1}\meta{token2}\marg{default}\marg{continue}]
  Like \cs{MRTutil@Oarg}, but the optional argument is delimited by
  \meta{token1} and \meta{token2}. So \bverb|\MRTutil@Darg<>{}\foo| will check
  whether there is an optional argument delimited by \texttt{<>} and if there is
  none will use an empty one. The result is provided to \cs{foo}.
\end{describemacro}

\begin{describemacro}{MRTutil@darg}%
  [\meta{token1}\meta{token2}\marg{continue}]
  Like \cs{MRTutil@darg}, but provides a special marker if there is no optional
  argument. You can check whether the special marker was provided with
  \cs{MRTutil@ifmark}. 
\end{describemacro}

\begin{describemacro}{MRTutil@Earg}[\meta{token}\marg{default}\marg{continue}]
  checks for a following optional argument which is indicated by a leading
  \meta{token}. If there is a \meta{token} following, the following argument
  will be grabbed according to \TeX's rules (so a single token might follow or a
  group delimited by |{}|). If there is no \meta{token} following it provides
  \meta{default}. \meta{continue} will be executed after the argument has been
  parsed. The value of the optional argument will be provided to \meta{continue}
  as it is present in the input stream (so if it is braced there will be braces,
  if it's just a single token, there will be only a single token).
\end{describemacro}

\begin{describemacro}{MRTutil@earg}[\meta{token}\marg{continue}]
  Like \cs{MRTutil@Earg}, but if there is no \meta{token} following instead of a
  default, a special marker will be provided. You can check whether the special
  marker was provided with \cs{MRTutil@ifmark}.
\end{describemacro}

\begin{describemacro}{MRTutil@targ}[\meta{token}\marg{continue}]
  checks for a following \meta{token}. If that \meta{token} follows it will be
  gobbled and \meta{continue} will get a special marker as its argument. Else
  another marker will be provided that would result in |false| in
  \cs{MRTutil@ifmark}. This is similar to \LaTeXe's \cs{@ifstar}.
\end{describemacro}

\begin{describemacro}{MRTutil@ifmark}[\marg{test}\marg{true}\marg{false}]
  Tests whether \meta{test} is the special marker provided by \cs{MRTutil@oarg}
  and similar. If so expands to \meta{true}, if not it expands to \meta{false}.
\end{describemacro}

\begin{describemacro}{MRTutil@defOargpair}
  [\marg{cs1}\marg{cs2}\meta{token1}\meta{token2}]
  This defines \meta{cs1} to be similar to \cs{MRTutil@Oarg} and \meta{cs2} to
  be similar to \cs{MRTutil@oarg}, but using \meta{token1} and \meta{token2} as
  delimiters. The resulting macros should be faster than \cs{MRTutil@Darg} and
  \cs{MRTutil@darg}. Additionally to \meta{cs1} and \meta{cs2} a macro called
  \meta{\textbackslash cs1} (with an additional backslash in its name) will be
  defined that does the argument grabbing. \meta{cs1} and \meta{cs2} should
  include the leading backslash.
\end{describemacro}

\begin{describemacro}{MRTutil@defOarg,MRTutil@defoarg}
  [\marg{cs}\meta{token1}\meta{token2}]
  These can be used if only one of the equivalents of \cs{MRTutil@Oarg} or
  \cs{MRTutil@oarg} is needed. Additionally to \meta{cs} a macro called
  \meta{\textbackslash cs} (with an additional backslash in its name) will be
  defined that does the argument grabbing. \meta{cs} should include the leading
  backslash. There is a usage example in \autoref{sec:util:example}.
\end{describemacro}
%=<<

\subsection{Example}\label{sec:util:example}%>>=
In the following example we'll define a macro that takes two optional arguments
-- one delimited by |[]| and one delimited by |<>| -- and a mandatory one. We'll
need one auxiliary macro per optional argument.

\begingroup
First we define the front facing macro named \cs{ourmacro}. It is defined
\cs{protected}, because it isn't expandable since \cs{MRTutil@oarg} isn't.
\DoAndPrint
{\makeatletter
\MRTutil@def[\protected]\ourmacro{\MRTutil@oarg{\ourmacro@a}}}
\noindent
Next we define the second step that looks for the second optional argument,
again defined \cs{protected}, but this time also \cs{long}, because it takes
arguments and those might contain an explicit or implicit \cs{par}. We pipe
through the first optional argument. Since we think that \cs{ourmacro} is gonna
be used pretty often we want the argument grabbing for the second optional
argument to be fast, so instead of using \cs{MRTutil@darg<>} we define our own
test macro \cs{ourmacro@opt} (the speed gain is really small though).
\DoAndPrint
{\MRTutil@defoarg\ourmacro@opt<>
\MRTutil@def[\protected\long]\ourmacro@a#1{\ourmacro@opt{\ourmacro@b{#1}}}}
\noindent
The last step is the one which takes both optional arguments and the mandatory
one. This one doesn't have to be defined \cs{protected}, because it is
expandable.
\begin{verbatim}
\MRTutil@def[\long]\ourmacro@b#1#2#3%
  {%
    \MRTutil@ifmark{#1}
      {No 1st optional argument provided}
      {1st optional argument: #1}%
    \par
    \MRTutil@ifmark{#2}
      {No 2nd optional argument provided}
      {2nd optional argument: #2}%
    \par
    Mandatory argument: #3%
  }
\makeatother
\end{verbatim}
\MRTutil@def[\long]\ourmacro@b#1#2#3%
  {%
    \MRTutil@ifmark{#1}
      {No 1st optional argument provided}
      {1st optional argument: #1}%
    \par
    \MRTutil@ifmark{#2}
      {No 2nd optional argument provided}
      {2nd optional argument: #2}%
    \par
    Mandatory argument: #3%
  }
As you can see, the macro doesn't do anything special, it just lists its
arguments. A few usage examples are shown in \autoref{tab:util:ourmacro}.

\MRTutil@def\ourmacrocell{\setstretch{1}\expandafter\ourmacro\@firstofone}%
\makeatother
\noindent
\begin{MRTtable}
  [
    ,pre=
      {%
        \setbox0\hbox{No 2nd optional argument provided}%
        \def\mc{\multicolumn{1}}%
      }
    ,caption=Usage examples of \cs{ourmacro}
    ,label=tab:util:ourmacro
    ,col=
      {
        >{\ttfamily\cs{ourmacro}\collectcell\detokenize}l<{\endcollectcell}
        >{\collectcell\ourmacrocell}p{\wd0}<{\endcollectcell}
      }
  ]
  \mc{l}{Macro call} & \mc{p{\wd0}}{Output} \\
  {baz}              & {baz}                \\[0pt]
  [foo]{baz}         & [foo]{baz}           \\
  <bar>{baz}         & <bar>{baz}           \\[0pt]
  [foo]<bar>{baz}    & [foo]<bar>{baz}      \\
\end{MRTtable}
\endgroup
%=<<

% vim: ft=tex fdm=marker fmr=>>=,=<< sw=2 ts=2 tw=80
