\chapter{The \cls{MRTalone} class}
The \cls{standalone} version of \cls{MRTthesis}. The aim is to provide a class
to produce simple \LaTeX\ based images which match the look of \cls{MRTthesis}.

\section{Options and Setup Files}\label{sec:alone:files}%>>>
\cls{MRTalone} allows you to use a shared options file and a shared setup file
for your project. The files get sourced if they are available.

The shared options might be placed into the file \file{./MRTalone.option.tex}.
This file can include any number of \cs{MRTaloneOptions} calls. It is loaded in
the midst of the class file. See \autoref{sec:alone:macros} for a description of
\cs{MRTaloneOptions} and \autoref{sec:alone:options} for the available options.

The additional setup file should be \file{./MRTalone.setup.tex}. It is sourced
at the end of the class and might contain any valid \LaTeX\ code, of course
including some \cs{MRTaloneSetup} instructions. See \autoref{sec:alone:macros}
for \cs{MRTaloneSetup} and \autoref{sec:alone:setup} for the available setup
options.
%<<<

\section{Options}%>>>
\subsection{Load time options}\label{sec:alone:options}%>>>
The class features a few load time options.
\begin{describeopt}{longtable}
  Is forwarded to \pkg{MRTtab} see its description in \autoref{sec:tab:options}.
\end{describeopt}
\begin{describeopt}{mathsizes}
  Opposite of \opt{no mathsizes}. If used (which it by default is) the maths
  sizes are set according to the MS Word template. Note that those weren't set
  by \pkg{mrtarbeit} and if you alter the default font size won't be set.
\end{describeopt}
\begin{describeopt}{minimal}
  If this option is passed some packages are not loaded and therefore related
  configurations not set. See \autoref{sec:alone:dep}.
\end{describeopt}
\begin{describeopt}{no mathsizes}
  Opposite of \opt{mathsizes}. If used the maths sizes are not changed from the
  defaults of \cls{scrreprt}.
\end{describeopt}
\begin{describeopt}{tikzunderline, tUline}
  This option is forwarded to \pkg{MRTwuline}. See its description in
  \autoref{sec:wuline:options}.
\end{describeopt}
\begin{describeopt}{british,english,UKenglish}
  If used the document will be using the \opt{british} definition of
  \pkg{babel}. Many strings used in the package will be in English, but some
  might be missed out. If you find any of which you think should be translated,
  please contact me as described in \autoref{sec:bugs}. English simplified (US)
  is not supported by the class.
\end{describeopt}
The following options (and their values) will be forwarded to \pkg{MRTfonts}
(see \autoref{sec:fonts:options} for their description):
\begingroup
\multicolsep=0pt
\premulticols=0pt
\begin{multicols}{4}
\begin{itemize}
  \item \opt{sfacc}
  \item \opt{font}
  \item \opt{serif font}
  \item \opt{mono font}
  \item \opt{new maths}
  \item \opt{scale maths}
  \item \opt{scale macro}
  \item \opt{alt l}
  \item \opt{std l}
  \item \opt{mathsizes}
  \item \opt{no mathsizes}
  \item \opt{pmb}
\end{itemize}
\end{multicols}
\endgroup


\noindent
Every other given option will be passed on to \cls{standalone}.
%<<<
\subsection{Setup options}\label{sec:alone:setup}%>>>
The following options are accessible with \cs{MRTaloneSetup}.
\begin{describeopt}{caption above}
  Is forwarded to \pkg{MRTtab} and its \cs{MRTtabSetup}. See its description in
  \autoref{sec:tab:options:setup}.
\end{describeopt}
\begin{describeopt}{caption below}
  Is forwarded to \pkg{MRTtab} and its \cs{MRTtabSetup}. See its description in
  \autoref{sec:tab:options:setup}.
\end{describeopt}
\begin{describeopt}{stretch caption, stretch cap}[\meta{float}]
  Currently does nothing.
\end{describeopt}
\begin{describeopt}{stretch tabular, stretch tab}[\meta{float}]
  Is forwarded to \pkg{MRTtab} and its \cs{MRTtabSetup}. See its description in
  \autoref{sec:tab:options:setup}.
\end{describeopt}
\begin{describeopt}{stretch text}[\meta{float}]
  Uses \cs{setstretch} to set a specific line spread in the document.
\end{describeopt}
\begin{describeopt}{stretches}[\meta{float}]
  Sets \opt{stretch cap}, \opt{stretch tab}, and \opt{stretch text} in one go.
  Initially set to \texttt{1.408}.
\end{describeopt}
%<<<
%<<<

\section{Macros}\label{sec:alone:macros}%>>>
\begin{describemacro}{MRTaloneSetup}[\marg{options}]%>>>
  You can use this macro to set the options listed in \autoref{sec:alone:setup}.
\end{describemacro}%<<<
\begin{describemacro}{MRTaloneOptions}[\marg{options}]%>>>
  You can use this macro to set the options listed in
  \autoref{sec:alone:options}. It is only available inside of the
  \file{./MRTalone.option.tex} file (see \autoref{sec:alone:files}).
\end{describemacro}%<<<
%<<<

\section{Dependencies}\label{sec:alone:dep}%>>>
The class is based on \cls{standalone}, therefore it naturally depends on that
and all its dependencies. Additional dependencies are:
\begin{multicols}{2}%>>>
  \begin{itemize}[leftmargin=10pt]
    \item \pkg{expl3}
    \item \pkg{xparse}
    \item \pkg{MRTtab} for which \opt{in text sep} is set to 0pt and the option
      \opt{no float} is set. Take a look at \autoref{sec:tab:options:setup} to
      see what those do.
    \item \pkg{MRTwuline}
    \item \pkg{MRTsfacc}
    \item \pkg{MRTfonts}
    \item \pkg{babel} [ngerman] or if \opt{british} is used with
      [main=british, ngerman]
    \item \pkg{setspace}
    \item \pkg{enumitem}
    \item \pkg{mathtools} with the \opt{fleqn} option
    \item \pkg{mathastext} with the \opt{defaultmathsizes} and \opt{italic}
      options
    \item \pkg{isomath}
    \item if the \opt{minimal} option is not used:
      \begin{itemize}
        \item \pkg{siunitx}
          [%
            detect-all,
            per-mode=reciprocal-positive-first%
          ]\\
          If \pkg{babel}'s \opt{british} is used [locale=UK] will be used, if
          \opt{ngerman} [locale=DE]. Additionally the \opt{range-phrase} will be
          set to either \texttt{to} or \texttt{bis} with spaces around it.
      \end{itemize}
  \end{itemize}
\end{multicols}%<<<
%<<<
