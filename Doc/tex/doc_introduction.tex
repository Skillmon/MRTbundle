\chapter{Introduction}
This bundle provides three \LaTeX\ classes, one for theses and one for
presentations, which both aim to match the corresponding MS Office templates of
the Chair of Measurement and Control Engineering (Lehrstuhl für Mess"~~und
Regeltechnik; MRT) of the University of Bayreuth, hence the name. Along the two
major classes \cls{MRTthesis} and \cls{MRTbeam} there is another class to
create stand alone images and minor auxiliary packages contained in this
distribution.

The classes are originally created for use with \hologo{pdfLaTeX} and give the
best results with it. This is caused by the available fonts. The classes were
created for use with the \pkg{helvet} font which is not a good choice for
\luaxelatex*.

This bundle makes no claim to be complete, comprehensive, or correct. For
formatting errors I don't take any responsibility. Each author takes full
liability for his work and its formatting.

You're allowed to share this work with fellow students working at the MRT,
though official distribution channels might be better suited as they assure up
to date versions.

I'd feel guilty distributing this bundle without saying the following: I'm not
responsible for the overall look of this. I tried to match the Word template of
the institution where possible and as a result, this is non-optimal typography,
in my humble opinion.

Of course this documentation is created with one of the provided classes, namely
\cls{MRTthesis}, in use.

If you're not yet familiar with \LaTeX\ you should stop reading at this point
(meaning the end of this paragraph) and either read a \emph{good} and
\emph{up-to-date} introduction to \LaTeX\ and afterwards read on or use MS Word
for your thesis. Personally I think the time reading an introduction in order to
use \LaTeX\ is well spend, but there certainly are different opinions on that --
unfortunately opinions are prone to be biased, mine is no exception. There is
a great short tutorial (just enough to get you started) online which is kept up
to date by the \LaTeX\ project team: \url{https://www.learnlatex.org}. Another
viable introduction is lshort which is available in several languages at the
following link: \url{https://www.ctan.org/pkg/lshort}

\section{Feature Requests and Bug Reports}\label{sec:bugs}
You can request features or report bugs at gitlass, where the most up to date
version is hosted:
\url{https://gitlass.de/jonathan/MRTbundle}. Alternatively if you have a github
account you can use the issue tracker there:
\url{https://github.com/Skillmon/MRTbundle/issues}, the repository is not
mirrored though, \emph{so the code there is out of date}.

You can request features or report bugs if you find some via
email, too:
\href{mailto:jspratte@yahoo.de?subject=MRTbundle -- bug report}
  {jspratte@yahoo.de}.
Please use a descriptive subject containing ``MRTbundle'' (e.g. ``MRTbundle --
bug report'').

\section{Support and Questions}
I aimed to make this documentation as complete as possible and as short as
reasonable. This might have the effect that some things are explained too
briefly and are not comprehensible or that other things get lost between all the
other stuff. In cases like these you might have questions which I'd be happy to
answer. Since my spare time might be limited you should first try to understand
everything you have problems with using this documentation and perhaps taking a
look at the provided example documents.

If you have questions which you can't solve with the material provided and the
question is related to this bundle and not a general \LaTeX\ question, contact
me via email:
\href{mailto:jspratte@yahoo.de?subject=MRTbundle -- support}
  {jspratte@yahoo.de}.
Make sure the subject is descriptive. My answer might redirect you to other
resources if I don't think the issue is related to this bundle. You can test
whether your issue is related to this bundle by trying to recreate it using
another class and without any \pkg{MRT} packages. If that is possible, it is
\emph{not} related. If your query is stated somewhere in this documentation, my
answer might not be very kind (\!\emph{RTFM}).

Unrelated questions might be asked on \url{tex.stackexchange.com} (English) or
\url{texwelt.de} (German).\footnote{Funny thing, I might answer you there, too,
if I got time and your question isn't answered until then.} Since the vast
majority of people there don't have the packages and classes provided in this
bundle installed, make sure that your example code doesn't rely on those, so
that the \emph{volunteers} can actually help you.

\section{Individual Versions}
\docIndividualVersions
