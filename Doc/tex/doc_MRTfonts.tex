\chapter{The \pkg{MRTfonts} package}
\pkg{MRTfonts} loads the fonts as they are used by the classes of this bundle,
giving a uniform look to the documents.

\section{Options}\label{sec:fonts:options}
%>>=
\begin{describeopt}{mathsizes}
  Opposite of \opt{no mathsizes}. If used (which it by default is) the maths
  sizes are set according to the MS Word template. Note that those weren't set
  by \pkg{mrtarbeit} and if you alter the default font size won't be set.
\end{describeopt}
\begin{describeopt}{no mathsizes}
  Opposite of \opt{mathsizes}. If used the maths sizes are not changed from the
  defaults of the base class in use (so \cls{scrreprt}, \cls{standalone} or
  \cls{beamer}).
\end{describeopt}
\begin{describeopt}{sfacc}[\meta{choice}]
  \meta{choice} must be \opt{height} or \opt{list}. Sets the approach used by
  \pkg{MRTsfacc} (see \autoref{sec:sfacc}) and if \opt{list} is in use the shift
  list for \pkg{helvet} will also be loaded. If it is not specified the
  \opt{list} variant is used.
\end{describeopt}
\begin{describeopt}{scale macro}
  If you use this option the macro \cs{scalemath} will be available. It is no
  longer needed for \cs{altl} but if you really need it this option can provide
  downward compatibility.
\end{describeopt}
\begin{describeopt}{pmb}
  If you use this option the macro \cs{pmb} will be redefined to give, imho,
  better looking results. Of course, one shouldn't use \cs{pmb} anyway, but
  unfortunately the Greek letters used with \opt{new maths} don't feature bold
  glyphs, so if you want to use bold Greek letters, you'll have to use it. If
  \pkg{bm} is loaded in the preamble its version of \textit{poor man's bold}
  will be redefined, too. The effect is that instead of three overlapping
  slightly shifted symbols a whooping fourteen will be used. Compare and guess
  which one is the original and which one is the altered version:\\
  \null\hfill
  \scalebox{2.5}{$\alpha$}
  \includegraphics[trim=1pt 1pt 1pt 0pt,clip,scale=2.5]
    {img/fonts_pmb_mine-alone}
  \includegraphics[trim=1pt 1pt 1pt 0pt,clip,scale=2.5]
    {img/fonts_pmb_orig-alone}
  \hfill\null
\end{describeopt}
\begin{describeopt}{alt l}
  If this option is used the letter |l| in maths will result in the same as the
  \cs{altl} macro (see \autoref{sec:fonts:macros}). This option is used by
  default.
\end{describeopt}
\begin{describeopt}{std l}
  If this option is used the letter |l| in maths will result in the same as the
  \cs{stdl} macro (see \autoref{sec:fonts:macros}).
\end{describeopt}
\begin{describeopt}{font}[\meta{font}]
  \emph{Deprecated.}
\end{describeopt}
\begin{describeopt}{serif font}[\meta{font}]
  \emph{Deprecated.}
\end{describeopt}
\begin{describeopt}{mono font}[\meta{font}]
  \emph{Deprecated.}
\end{describeopt}
\begin{describeopt}{new maths}[\meta{choice}]
  This is only available if you're using \hologo{pdfTeX}. With this you can
  specify whether some special maths fonts are loaded. The result looks closer
  to the MS Word template for Greek letters and operators.\\
  Available \meta{choice}s are |off| or |false| to turn this off, |on| or |true|
  to turn this on, and a valid float, to set the scale of the Greek letters and
  activate the feature.  By default |1.05| will be used.\\
  The number of usable maths fonts will be used exhaustively. \pkg{newpxmath}
  will be loaded with its options \opt{upint}, \opt{smallerops},
  \opt{nosymbolsc} and \opt{noamssymbols} to get operators and the like, the
  maths letters of \pkg{mathptmx} will be loaded with a scale factor to get the
  Greek letters.  The \pkg{lmodern} package will be loaded with the \opt{nomath}
  option, and \pkg{MRTlmscale} will not be loaded at all.\\
  Unfortunately \pkg{newpxmath} v1.401 from 2019-10-02 has a bug (that is
  already reported) in which the symbols of \cs{forall} and \cs{exists} aren't
  the correct ones (instead $\hslash$ and $\hbar$ are used). There might be
  other affected symbols, too, \pkg{MRTfonts} fixes these two for you.
\end{describeopt}
\begin{describeopt}{scale maths}[\meta{choice}]
  This is only available if you're using \hologo{pdfTeX}. With this you can
  specify whether the \pkg{MRTlmscale} package should be loaded. Available
  \meta{choice}s are no argument, resulting in \pkg{MRTlmscale} being used with
  its default, |on| or |true| resulting in the same, |off| or |false| resulting
  in \pkg{MRTlmscale} not being used, and any valid float, resulting in
  \pkg{MRTlmscale} being used with the specified float as its scale factor. See
  \autoref{sec:lmscale} for more about \pkg{MRTlmscale}. By default,
  \pkg{MRTlmscale} will be used with its default scale factor if \pkg{lmodern}
  is loaded with its maths fonts.
\end{describeopt}
%=<<

\section{Macros}\label{sec:fonts:macros}
%>>=
\begin{describemacro}{stdl}
  \emph{Only available in \hologo{pdfTeX}.}\\
  \cs{stdl} will result in the lower case |l| from the \pkg{helvet} font in
  maths (\bverb|$\stdl$| results in $\stdl$).
\end{describemacro}
\begin{describemacro}{altl}
  \emph{Only available in \hologo{pdfTeX}.}\\
  \cs{altl} provides an alternative lower case |l| for use in maths which is
  distinct from an upper case |I|. Compare: $\stdl I$ (that is |$\stdl I$|) and
  $\altl I$ (that is |$\altl I$|). There is no bold version of \cs{altl}
  provided by the package, nor any other maths alphabet version. Instead the
  standard fonts will be the ones used there.
\end{describemacro}
\begin{describemacro}{arev}[\marg{symbols}]
  \emph{Only available in \hologo{pdfTeX}.}\\
  This is another maths font (similar to \cs{mathbf} or \cs{mathcal}), that will
  use the maths font of \pkg{arevmath}, from which \cs{altl} is taken. Take a
  look at the following (the first group uses the standard maths fonts, the
  second is typeset using \cs{arev}):\\[\parskip]
  $abcdefghijk\stdl mnopqrstuvwxyzABCDEFGHIJKLMNOPQRSTUVWXYZ$\\
  $
    0123456789
    \alpha\beta\gamma\delta\varepsilon\epsilon\zeta\eta\theta\iota\kappa\lambda
    \mu\nu\xi\pi\rho\sigma\varsigma\tau\upsilon\phi\chi\psi\omega
    \Gamma\Delta\Theta\Lambda\Xi\Sigma\Phi\Psi\Omega
  $\\[\parskip]
  $\arev{abcdefghijklmnopqrstuvwxyzABCDEFGHIJKLMNOPQRSTUVWXYZ}$\\
  $\arev{
    0123456789
    \alpha\beta\gamma\delta\varepsilon\epsilon\zeta\eta\theta\iota\kappa\lambda
    \mu\nu\xi\pi\rho\sigma\varsigma\tau\upsilon\phi\chi\psi\omega
    \Gamma\Delta\Theta\Lambda\Xi\Sigma\Phi\Psi\Omega
  }$
\end{describemacro}
\begin{describemacro}{scalemath}[\marg{float}\marg{text}]
  This is a version of \cs{scalebox} to be used in maths. It is only available
  if you use the \opt{scale macro} option.
\end{describemacro}
%=<<

\section{Dependencies}
\begin{multicols}{2}%>>=
  \begin{itemize}[leftmargin=10pt]
    \item \pkg{expl3}
    \item \pkg{MRTif}
    \item \pkg{MRTutil}
    \item \pkg{MRTsfacc}
    \item If the \opt{scale macro} option is used \pkg{graphics}
    \item If \hologo{XeTeX} or \hologo{LuaTeX} are used
      \begin{itemize}[topsep=0pt]
        \item \pkg{fontspec} [no-math] and additionally
          \bverb|\defaultfontfeatures{Ligatures=TeX}|
        \item the fonts \texttt{TeX Gyre Heros}, \texttt{Latin Mod\-ern Roman}
          and \texttt{Latin Modern Mono}.
      \end{itemize}
      else
      \begin{itemize}[topsep=0pt]
        \item \pkg{fontenc} [T1]
        \item \pkg{helvet}
        \item If \opt{new maths} is used
          \begin{itemize}[topsep=0pt]
            \item \pkg{lmodern} [nomath]
            \item \pkg{newpxmath} [upint, smallerops, nosymbolsc, noamssymbols]
            \item The |ztmcm| font
          \end{itemize}
          else
          \begin{itemize}[topsep=0pt]
            \item \pkg{lmodern}
            \item \pkg{MRTlmscale} dependent on the \opt{scale maths} key
          \end{itemize}
      \end{itemize}
    \item \pkg{mathastext} [italic, defaultmathsizes]
    \item \pkg{isomath}
  \end{itemize}
\end{multicols}%=<<

% vim: ft=tex fdm=marker fmr=>>=,=<< sw=2 ts=2 tw=80
