\chapter{The \pkg{MRTfonts} package}
\pkg{MRTfonts} loads the fonts as they are used by the classes of this bundle,
giving a uniform look to the documents.

\section{Options}\label{sec:fonts:options}
\begin{describeopt}{sfacc}[\meta{choice}]
  \meta{choice} must be \opt{height} or \opt{list}. Sets the approach used by
  \pkg{MRTsfacc} (see \autoref{sec:sfacc}) and if \opt{list} is in use the shift
  list for \pkg{helvet} will also be loaded. If it is not specified the
  \opt{list} variant is used.
\end{describeopt}
\begin{describeopt}{font}[\meta{font}]
  This is only available if you're using \luaxetex. With this you can set the
  used sans serif font, which will be used as the default font. It should be
  a font resembling Helvetica or Arial. Per default the font
  \bverb|TeX Gyre Heros| will be used.
\end{describeopt}
\begin{describeopt}{serif font}[\meta{font}]
  This is only available if you're using \luaxetex. With this you can set the
  used serif font. This isn't too important as the default fonts will suffice.
  You shouldn't have too much text with a Roman font anyway. Per default the
  font \bverb|Latin Modern Roman| will be used.
\end{describeopt}
\begin{describeopt}{mono font}[\meta{font}]
  This is only available if you're using \luaxetex. With this you can set the
  used mono font. This isn't too important as the default fonts will suffice.
  You shouldn't have too much text with a mono font anyway. Per default the
  font \bverb|Latin Modern Mono| will be used.
\end{describeopt}
\begin{describeopt}{new maths}[\meta{choice}]
  This is only available if you're using \hologo{pdfTeX}. With this you can
  specify whether some special maths fonts are loaded. The result looks closer
  to the MS Word template for Greek letters and operators.\\
  Available \meta{choice}s are |off| or |false| to turn this off, |on| or |true|
  to turn this on, and a valid float, to set the scale of the Greek letters and
  activate the feature.  By default |1.05| will be used.\\
  The number of usable fonts will be used exhaustively. The \pkg{newpxmath} will
  be loaded with its \opt{upint} and \opt{slantedGreek}, the math letters of
  \pkg{mathptmx} will be loaded with a scale factor. \pkg{bm} will be loaded so
  that you can use the bold symbols of the Latin Modern fonts for Greek letters.
\end{describeopt}
\begin{describeopt}{scale maths}[\meta{choice}]
  This is only available if you're using \hologo{pdfTeX}. With this you can
  specify whether the \pkg{MRTlmscale} package should be loaded. Available
  \meta{choice}s are no argument, resulting in \pkg{MRTlmscale} being used with
  its default, |on| or |true| resulting in the same, |off| or |false| resulting
  in \pkg{MRTlmscale} not being used, and any valid float, resulting in
  \pkg{MRTlmscale} being used with the specified float as its scale factor. See
  \autoref{sec:lmscale} for more about \pkg{MRTlmscale}. By default,
  \pkg{MRTlmscale} will be used with its default scale factor.
\end{describeopt}

\section{Macros}
\begin{describemacro}{altl}
  \cs{altl} provides an alternative lower case |l| for use in maths which is
  distinct from an upper case |I|. Compare: $lI$ (that is |$lI$|) and $\altl I$
  (that is |$\altl I$|).
\end{describemacro}
\begin{describemacro}{altlUnscaled}
  \cs{altl} is a scaled version of the letter it uses. \cs{altlUnscaled} is the
  unscaled original version. Compare: $\altl I\altlUnscaled$ (that is
  |$\altl I\altlUnscaled$|). Don't change its definition, as it is used by
  \cs{altl} internally.
\end{describemacro}
\begin{describemacro}{scalemath}[\marg{float}\marg{text}]
  This is a version of \cs{scalebox} to be used in maths. It is used by
  \cs{altl} to scale down \cs{altlUnscaled}. 
\end{describemacro}

\section{Dependencies}
\begin{multicols}{2}%>>=
  \begin{itemize}[leftmargin=10pt]
    \item \pkg{expl3}
    \item \pkg{MRTif}
    \item \pkg{MRTutil}
    \item \pkg{graphics}
    \item If \hologo{XeTeX} or \hologo{LuaTeX} are used
      \begin{itemize}[topsep=0pt]
        \item \pkg{fontspec} and \bverb|\defaultfontfeatures{Ligatures=TeX}|
        \item the fonts \bverb|TeX Gyre Heros|, \bverb|Latin Modern Roman| and
          \bverb|Latin Modern Mono|, dependent on the |font| keys.
      \end{itemize}
      else
      \begin{itemize}[topsep=0pt]
        \item \pkg{fontenc} [T1]
        \item \pkg{inputenc} [utf8]
        \item \pkg{lmodern}
        \item \pkg{helvet}
        \item \pkg{MRTlmscale} dependent on the \bverb|scale maths| key.
        \item If \opt{new maths} is used
          \begin{itemize}
            \item \pkg{bm}
            \item \pkg{newpxmath} [upint, slantedGreek]
            \item The |ztmcm| font
          \end{itemize}
      \end{itemize}
  \end{itemize}
\end{multicols}%=<<

% vim: ft=tex fdm=marker fmr=>>=,=<< sw=2 ts=2 tw=80
