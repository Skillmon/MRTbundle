\chapter{The \pkg{MRTsfacc} package}\label{sec:sfacc}
This package is provided to remedy an issue related with sans serif maths, to be
more precise to fix the placement of \cs{mathaccentV}, which is internally used
by macros such as \cs{hat} and \cs{dot} with \pkg{amsmath} loaded. It is
therefore loaded by all three, \cls{MRTthesis}, \cls{MRTbeam} and
\cls{MRTalone}. The \cls{beamer} class provides a fix for the same issue which
is unfortunately only working for \cls{beamer}'s default font by fixing the font
metrics (and as far as I know only works with \hologo{pdfLaTeX}).

\cls{MRTsfacc} has two different approaches by patching \cs{mathaccentV} to move
the accent horizontally -- either depending on the height of the accented
character or a defined offset in a list of possible arguments.

The package is designed with \pkg{mathastext} with the \opt{italics} option in
mind. It might work for other sans serif maths solutions as well. It requires
\pkg{amsmath} to be loaded. It is incompatible with the \pkg{accents} package,
as that one changes the accents to no longer use \cs{mathaccentV} internally.

Independent on the used approach the accent macros check whether their argument
is one meeting a special criterion (a character of category 11 or a known
element). Furthermore both versions should detect whether the argument is just
another accent macro nested so that in \bverb|\dot{\bar{a}}| the \cs{dot} would
still find the |a| as a known argument. This nested usage works only if the
nested macro uses \cs{mathaccentV} internally and each level of nesting is an
exact match of the approach's criterion or does only contain two tokens or
groups (so in above example the \cs{bar} and the |{a}|) with the first one being
a \cs{mathaccentV} using macro.

\section{Options}%>>=
The package has the following options:
\begin{describeopt}{height}
  If this option is used the offset is dependent on the height of the accented
  character. Read the description in \autoref{sec:sfacc:height}.
\end{describeopt}
\begin{describeopt}{list}
  If this option is used the offset is defined by a list of known arguments.
  Read the description in \autoref{sec:sfacc:list}.
\end{describeopt}
\begin{describeopt}{notest}
  By default the package does test whether the definition of \cs{mathaccentV}
  meets the known definition from the \pkg{amsmath} package. If something does
  redefine \cs{mathaccentV} or the definition has changed but you're sure that
  \pkg{MRTsfacc} still works with (it redefines it anyway) you can deactivate
  that test with this option. If \pkg{amsmath}'s definition of \cs{mathaccentV}
  has changed, please contact the maintainer as described in \autoref{sec:bugs}.
\end{describeopt}

Every other option is passed on to \cs{MRTsfaccSet}, its description is included
in \autoref{sec:sfacc:height:macros}. This will have no effect if the \opt{list}
option is used.
%=<<

\section{\opt{height} Variant}\label{sec:sfacc:height}%>>=
This variant checks whether the argument is a single character with category
code 11. If this test does not return true, the shift isn't applied.

\autoref{tab:sfacc:height} shows the results of this approach. While this
approach is easier to adapt to other fonts -- one has to change only one
parameter -- it is always a compromise trying to match every character as good
as possible.

\begin{MRTtable}% >>=
  [
    ,env={}
    ,label=tab:sfacc:height
    ,caption=
      {%
        Comparison of shifted accents against original placement with the use of
        the \opt{height} variant.%
      }
  ]
  \includegraphics{img/sfacc_height_table-alone.pdf}
\end{MRTtable}% =<<

\subsection{Macros}\label{sec:sfacc:height:macros}%>>=
\begin{describemacro}{<accent>}[\meta{*/!}\marg{arg}]%>>=
  \cs{<accent>} can be any of the maths accent macros using \cs{mathaccentV}
  internally (e.g. \cs{bar}, \cs{dot}, etc.).\\
  The \meta{*/!} can either be \texttt{*} or \texttt{!}\@ or omitted entirely.
  If the starred version is used, the shift is enforced regardless of the
  argument, if the exclamation mark is given it is prohibited.
\end{describemacro}%=<<

\begin{describemacro}{MRTsfaccSet}[\marg{float}]%>>=
  The shift width depends on a multiple of the box's height. The multiple can be
  set with this macro and should be a valid float. This is tested using
  \cs{MRTifFloatTF}. The package default for this share is
  \makeatletter\texttt{\MRTsfacc@share}\makeatother.
\end{describemacro}%=<<
%=<<
%=<<

\section{\opt{list} Variant}\label{sec:sfacc:list}%>>=
This variant checks whether the argument is a known element from a list in which
the offset is defined in the unit of \texttt{mu}.

It has the advantage that you can define individual offsets for every argument.
In addition not only characters can be added to the list but almost arbitrary
stuff. The drawback is that everything has to be added that you want to be
recognized.
\autoref{tab:sfacc:list} shows the results of this approach.

\begin{MRTtable}% >>=
  [
    ,env={}
    ,label=tab:sfacc:list
    ,caption=
      {%
        Comparison of shifted accents against original placement with the use of
        the \opt{list} variant.%
      }%
  ]
  \includegraphics{img/sfacc_list_table-alone.pdf}
\end{MRTtable}% =<<

\subsection{Macros}
\begin{describemacro}{<accent>}[\oarg{opt}\marg{arg}]%>>=
  \cs{<accent>} can be any of the maths accent macros using \cs{mathaccentV}
  internally (e.g. \cs{bar}, \cs{dot}, etc.).\\
  \meta{opt} can either be a defined element from the list or a length in the
  unit of \texttt{mu}. So with \texttt{foo} a defined list element, both
  \bverb|\hat[foo]{bar}| and \bverb|\hat[4mu]{bar}| would be valid. If
  \meta{opt} is a known element the offset of that element is used regardless of
  the given \meta{arg} else (if it is used) the given length is used as the
  offset. If the optional argument isn't used at all, it'll be tested whether
  \meta{arg} is a known element and if so the appropriate offset will be used.
  Else no offset will be applied.
\end{describemacro}%=<<

\begin{describemacro}{MRTsfaccShift}[\marg{element}\marg{shift}]%>>=
  Adds \meta{element} to the list of known arguments and saves \meta{shift} for
  it. If \meta{element} is already known it'll get redefined. \meta{shift} has
  to be given in \texttt{mu}.
\end{describemacro}%=<<

\begin{describemacro}{MRTsfaccShiftLet}%>>=
  [\marg{element1}\marg{element2}]
  Adds \meta{element1} to the list of known arguments and defines the
  offset to be the one currently used by \meta{element2}.
  \meta{element2} has to be known, if it isn't an error will be thrown.
\end{describemacro}%=<<

\begin{describemacro}{MRTsfaccLoadShiftList}[\marg{list}]
  The package comes with definitions for some fonts (see
  \autoref{tab:sfacc:listfiles}). With this macro you can load them. If you
  define a list for a font (or font combination) not listed in the table you
  might contact me as described in \autoref{sec:bugs} and I'll gladly add it
  to the package.
\end{describemacro}
%=<<

\begin{MRTtable}%>>=
  [
    ,col={>{\ttfamily}l >{\raggedright\arraybackslash}p{.55\linewidth}}
    ,cap={Available shift definition lists}
    ,label={tab:sfacc:listfiles}
  ]
    \normalfont List & To be used with\\
    helvet & \pkg{helvet} and \pkgWopt{mathastext}{italic,defaultmathsizes}
    \\
\end{MRTtable}%=<<

\section{Additional macros}
The package provides macros to use the accents used in text mode additionally in
maths. Since the placement proves somewhat difficult -- this might be caused by
the bundle's author's insufficient knowledge -- there is no really automated way
to do so with a few macros. Instead you can define macros which will produce a
symbol which is accented by one of the text accents.

\begin{describemacro}%>>=
  {%
    newsfhatmacro,defsfhatmacro,%
    newsfcheckmacro,defsfcheckmacro,%
    newsftildemacro,defsftildemacro,%
    newsfacutemacro,defsfacutemacro,%
    newsfgravemacro,defsfgravemacro,%
    newsfdotmacro,defsfdotmacro,%
    newsfddotmacro,defsfddotmacro,%
    newsfbrevemacro,defsfbrevemacro,%
    newsfbarmacro,defsfbarmacro%
  }%
  [%
    \sarg{horizontal}\oarg{vertical}\marg{cs}\oarg{type}\hspace{0pt}%
    \marg{symbol}%
  ]
  The difference between the \cs{def...} and the \cs{new...} variant is that
  the former will not check whether the macro \meta{cs} is already defined or
  not. With these macros you can locally create a \meta{cs} that gets displayed
  as \meta{symbol} with an accent based on the text font's variant of the
  accents.  \texttt{hat} uses \cs{^}, \texttt{check} uses \cs{v}, \texttt{tilde}
  uses \cs{~}, \texttt{acute} uses \cs{'}, \texttt{grave} uses \cs{`},
  \texttt{dot} uses \cs{.}, \texttt{ddot} uses \cs{"}, \texttt{breve} uses
  \cs{u} and \texttt{bar} uses \cs{=}.\\
  You can control the horizontal positioning of the accent using
  \meta{horizontal}, which should be a length in |mu|. If you don't provide
  \meta{horizontal} the offset will be determined based on the rules of the used
  variant (see \autoref{sec:sfacc:height} and \autoref{sec:sfacc:list}).
  \meta{vertical} specifies the vertical shift of the accent and should be given
  in |ex|. If \meta{vertical} is not given nothing special will happen (this
  might change in the future -- for now it is best if you specify |0ex| if you
  don't want to change the accents vertical placement).\\
  \meta{type} is the math atom type to be used for the newly created \meta{cs}.
  You could use \cs{mathord}, \cs{mathop}, \cs{mathbin}, \cs{mathrel},
  \cs{mathopen}, \cs{mathclose}, \cs{mathpunct}, \cs{mathinner}, or any other
  macro taking one argument.
\end{describemacro}%=<<

\begin{describemacro}{newsfaccmacro,defsfaccmacro}%>>=
  [%
    \sarg{horizontal}\oarg{vertical}\marg{cs}\oarg{type}\hspace{0pt}%
    \marg{accent}\marg{symbol}%
  ]
  This is a more general variant of \cs{newsfhatmacro} and the like. With this
  macro you can specify the macro responsible for typesetting the accent using
  the \meta{accent} argument. The specified \meta{accent} should take at most
  one argument and this one will be empty.
\end{describemacro}%=<<

The results of these macros heavily depend on the used font. For
\cls{MRTthesis}, using \hologo{pdfLaTeX}, the results don't look too bad. For
example one can define a \cs{hateq}:

\begingroup
\defsfhatmacro[-.3ex]\hateq[\mathrel]=
\begin{verbatim}
\defsfhatmacro[-0.3ex]{\hateq}[\mathrel]{=}
\end{verbatim}%
We use \cs{mathrel} since a |=| is a relation and should be spaced like that.
Additionally we move the accent down by |0.3ex|, which should give a good result
in this case. The following formula uses this \cs{hateq} definition,
|\mathrel{\hat{=}}| and the default |=| for comparison:
\begin{equation*}
  a\hateq a \mathrel{\hat{=}} a = a
\end{equation*}
\endgroup

Unfortunately these accents don't look too good in combination with Greek
letters (see for yourself: $\hat{\alpha}$ vs.
\begingroup\defsfhatmacro<.2mu>[0ex]\hatalpha\alpha$\hatalpha$\endgroup), and
one probably shouldn't mix the two types of accents in a document. The decision
which approach you use is up to you but you'd have to define a whole lot of
custom macros for every character you might want to use accented.

\section{Dependencies}
\pkg{MRTsfacc} loads the \pkg{MRTif} package and uses its tests
\cs{MRTifMathModeTF}, \cs{MRTifLetterGTF}, \cs{MRTifFloatTF},
\cs{MRTifStringsMatchXXTF} and \cs{MRTifTwoTokenTF}. It also depends on
\pkg{amsmath} being loaded. Additionally it uses the \pkg{MRTutil} package for
some of its macros' definitions.

% vim: ft=tex fdm=marker fmr=>>=,=<< sw=2 ts=2 tw=80
