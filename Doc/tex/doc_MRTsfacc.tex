\chapter{The \pkg{MRTsfacc} package}
This package is provided to remedy an issue related with sans serif maths, to be
more precise to fix the placement of \cs{mathaccentV}, which is internally used
by macros such as \cs{hat} and \cs{dot}. It is therefore loaded by both,
\cls{MRTthesis} and \cls{MRTbeam}. The \cls{beamer} class provides a fix for the
same issue which is unfortunately only working for \cls{beamer}'s default font
by fixing the font metrics.

\cls{MRTsfacc} has a different approach by patching \cs{mathaccentV} to move the
accent horizontally depending on the height of the accented character.
Furthermore it is tested whether the character is an alphabetic one by checking
the category code. If it is not an alphabetic character the shift isn't applied.

The package forwards any option to \cs{MRTsfaccSet}. \cs{MRTsfaccSet} tests
whether its argument is a valid float using \cs{MRTifFloatTF} and sets the share
of the letters' height which is used as the horizontal displacement. So you can
either use the option during load time to set the share or \cs{MRTsfaccSet} at
any later point. The default share is
\makeatletter\texttt{\MRTsfacc@shift}\makeatother.

One can use \texttt{*} to enforce the shift and \texttt{!}\@ to enforce the
omitting of that shift. Consider the following example:
\verb$\hat!{m}$ produces $\hat!{m}$, \verb$\hat*{m}$ produces $\hat*{m}$, and
\verb$\hat{m}$ produces $\hat{m}$, which is the same as \verb$\hat*{m}$ since
\texttt{m} has by default the category code of an alphabetic character.

The tokens \texttt{*} and \texttt{!}\@ must not be enclosed by braces if you
want to specify the behaviour of \cs{mathaccentV}, so \verb$\hat{*}\hat{!}$
results in $\hat{*}\hat{!}$.

\begin{MRTtable}
  [
    col=
      {
        *2{>{\collectcell\hatex}c<{\endcollectcell}}
        *2{>{\collectcell\hatst}c<{\endcollectcell}}
        >{\hskip4\tabcolsep}l
        *2{>{\collectcell\hatex}c<{\endcollectcell}}
        *2{>{\collectcell\hatst}c<{\endcollectcell}}
      }
    ,cap={Comparison of shifted accents against original placement}
    ,label=tab:sfacc
    ,pre=
      {%
        \newcommand\hatex[1]{$\hat!{#1}$}%
        \newcommand\hatst[1]{$\hat*{#1}$}%
      }
    ,pos=!bp
  ]
  \multicolumn{2}{c}{original} & \multicolumn{2}{c}{shifted} &&
  \multicolumn{2}{c}{original} & \multicolumn{2}{c}{shifted} \\
  
    a & A & a & A   &&   n & N & n & N \\
    b & B & b & B   &&   o & O & o & O \\
    c & C & c & C   &&   p & P & p & P \\
    d & D & d & D   &&   q & Q & q & Q \\
    e & E & e & E   &&   r & R & r & R \\
    f & F & f & F   &&   s & S & s & S \\
    g & G & g & G   &&   t & T & t & T \\
    h & H & h & H   &&   u & U & u & U \\
    i & I & i & I   &&   v & V & v & V \\
    j & J & j & J   &&   w & W & w & W \\
    k & K & k & K   &&   x & X & x & X \\
    l & L & l & L   &&   y & Y & y & Y \\
    m & M & m & M   &&   z & Z & z & Z \\
\end{MRTtable}

The resulting \cs{mathaccentV} macro is not expandable and therefore created
\cs{protected}. You can take a look at the results in \autoref{tab:sfacc}.

\section{Dependencies}
\pkg{MRTsfacc} loads the \pkg{MRTif} package and uses its tests
\cs{MRTifLetterGTF} and \cs{MRTifFloatTF}.
