\chapter{The \pkg{MRTsfacc} package}
This package is provided to remedy an issue related with sans serif maths, to be
more precise to fix the placement of \cs{mathaccentV}, which is internally used
by macros such as \cs{hat} and \cs{dot}. It is therefore loaded by both,
\cls{MRTthesis} and \cls{MRTbeam}. The \cls{beamer} class provides a fix for the
same issue which is unfortunately only working for \cls{beamer}'s default font
by fixing the font metrics.

\cls{MRTsfacc} has a different approach by patching \cs{mathaccentV} to move the
accent horizontally depending on the height of the accented character.
Furthermore it is tested whether the character is an alphabetic one by checking
the category code. If it is not an alphabetic character the shift isn't applied.

You can specify the shift based on the multiple of the height as the package
option. This is the only option the package takes and is not checked to be a
valid float. The default is \makeatletter\texttt{\MRTsfacc@shift}\makeatother.
Alternatively you can use \cs{MRTsfaccSet} to set the coefficient.

One can use \texttt{*} to enforce the shift and \texttt{!}\@ to enforce the
omitting of that shift. Consider the following example:
\verb$\hat!{m}$ produces $\hat!{m}$, \verb$\hat*{m}$ produces $\hat*{m}$, and
\verb$\hat{m}$ produces $\hat{m}$, which is the same as \verb$\hat*{m}$ since
\texttt{m} has by default the category code of an alphabetic character.

The tokens \texttt{*} and \texttt{!}\@ must not be enclosed by braces if you
want to specify the behaviour of \cs{mathaccentV}, so \verb$\hat{*}\hat{!}$
results in $\hat{*}\hat{!}$.

\begin{table}[!bp]
  \centering
  \setstretch{1.408}%
  \newcommand\hatex[1]{$\hat!{#1}$}%
  \newcommand\hatst[1]{$\hat*{#1}$}%
  \def\multfillcnt{6}%
  \hfillmult{\multfillcnt}\begin{MRTtabular}[][1]
    {%
      *2{>{\collectcell\hatex}c<{\endcollectcell}}
      *2{>{\collectcell\hatst}c<{\endcollectcell}}
    }
    \multicolumn{2}{c}{original} & \multicolumn{2}{c}{shifted} \\ 
    a & A & a & A \\
    b & B & b & B \\
    c & C & c & C \\
    d & D & d & D \\
    e & E & e & E \\
    f & F & f & F \\
    g & G & g & G \\
    h & H & h & H \\
    i & I & i & I \\
    j & J & j & J \\
    k & K & k & K \\
    l & L & l & L \\
    m & M & m & M \\
  \end{MRTtabular}\hfill
  \begin{MRTtabular}[][1]
    {%
      *2{>{\collectcell\hatex}c<{\endcollectcell}}
      *2{>{\collectcell\hatst}c<{\endcollectcell}}
    }
    \multicolumn{2}{c}{original} & \multicolumn{2}{c}{shifted} \\ 
    n & N & n & N \\
    o & O & o & O \\
    p & P & p & P \\
    q & Q & q & Q \\
    r & R & r & R \\
    s & S & s & S \\
    t & T & t & T \\
    u & U & u & U \\
    v & V & v & V \\
    w & W & w & W \\
    x & X & x & X \\
    y & Y & y & Y \\
    z & Z & z & Z \\
  \end{MRTtabular}\hfillmult{\multfillcnt}%
  \caption{Comparison of shifted accents against original placement}
  \label{tab:sfacc}
\end{table}

The resulting \cs{mathaccentV} macro is not expandable and therefore created
\cs{protected}. You can take a look at the results at \autoref{tab:sfacc}.

