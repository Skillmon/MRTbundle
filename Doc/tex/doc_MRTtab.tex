\chapter{The \pkg{MRTtab} package}
\pkg{MRTtab} provides means to typeset tables in a style similar to the ones in
the scripts of the MRT. This includes:
\begin{itemize}
  \item delimited by horizontal rules on top and below
  \item head rows are light grey and delimited by horizontal rules
  \item all horizontal rules have the same thickness
  \item no vertical rules (though not enforced)
\end{itemize}

The package provides an environment similar to \env{tabular}
(\autoref{sec:tab:tabular}), an enhanced version of \cs{cline}
(\autoref{sec:tab:cline}), and an environment to typeset displayed tables with
many options available (\autoref{sec:tab:table}).

\section{The \env{MRTtabular} environment}\label{sec:tab:tabular}%>>=
The \env{MRTtabular} environment calls a patched \env{tabular} environment. The
following differences exist:
\begin{itemize}
  \item a hook is provided at the beginning and the end of each line 
  \item above and below of it a \cs{hline} is placed
  \item it has an additional optional argument specifying the number of rows to
    be formatted as head rows.
  \item you can access the current row number
  \item automatic application of a stretch factor based on the \opt{stretch
    tabular} key in \autoref{sec:tab:options:setup}.
\end{itemize}
Any \env{tabular} environments inside of an \env{MRTtabular} are ordinary
\env{tabular}s which neither have hooks nor row numbers. They might be affected
by an outer \cs{rowcolor} or similar, though.

An ordinary description as done with other environments in this documentation:
\begin{describeenv}{MRTtabular}%>>=
  [\oarg{valign}\marg{preamble}\oarg{head rows}]
  The first optional argument as well as the mandatory argument match the ones
  of a regular \env{tabular} environment. \meta{head rows} specifies how many
  rows at the beginning of the environment should be formatted as head rows. If
  \meta{head rows} is not specified, no head row will be formatted. No
  further markup is required for this formatting to take place. You should end
  your rows only with \texttt{\string\\} to make the hook mechanism work (on
  which the head row markup relies).
\end{describeenv}%=<<

\begin{describeenv}{MRTarray}%>>=
  [\oarg{valign}\marg{preamble}\oarg{head rows}]
  This is to \env{MRTtabular} what \env{array} is to \env{tabular}, so all in
  all the same but every cell is formatted as maths and it should be used only
  inside of a maths context. All the hooks from \env{MRTtabular} work here, too.
\end{describeenv}%=<<

\begin{describemacro}{head}[\marg{num}]%>>=
  Additionally to the optional argument of \env{MRTtabular} to set the first $n$
  rows as head rows, you can use \cs{head} to set the next \meta{num} rows as
  head rows. This does not only work at the beginning of the environment but
  anywhere you want. Alternatively you can use the macros described in
  \autoref{sec:tab:explicit}. It is not supported in other implementations of
  tables. If you use it in \env{MRTtable} it'll throw an undefined error if
  \env{MRTtabular} isn't used internally (e.g., when you use the
  \opt{long} option or an alternative inner \env{tabular}-like environment via
  the \opt{env} keys).
\end{describemacro}%=<<

\begin{describemacro}{MRTtabAddtoBoLHook}[\marg{content}]%>>=
  You can add \meta{content} to the Begin-of-Line hook with this macro. Bear in
  mind that the \meta{content} should be fully expandable and not produce any
  text, if you want to use stuff like \cs{multicolumn}, \cs{rowcolor}, or
  \cs{cline} at the beginning of the line -- as this hook will be executed prior
  to that and \cs{noalign} and \cs{omit} won't work in that case. If you need
  something unexpandable that doesn't produce text you can enclose it in
  \cs{noalign}. The addition is made locally.
\end{describemacro}%=<<

\begin{describemacro}{MRTtabClearBoLHook}%>>=
  Clears the Begin-of-Line hook locally.
\end{describemacro}%=<<

\begin{describemacro}{MRTtabAddtoEoLHook}[\marg{content}]%>>=
  You can also add \meta{content} to the End-of-Line hook. Here it should not
  matter whether the contents are expandable or not, as it is impossible that
  something follows in the same row which can't follow something unexpandable.
  The addition is made locally.
\end{describemacro}%=<<

\begin{describemacro}{MRTtabClearEoLHook}%>>=
  Clears the End-of-Line hook locally.
\end{describemacro}%=<<

\begin{describemacro}{MRTtabCurrentRow}%>>=
  Returns the current row number in an \env{MRTtabular} expandably.
\end{describemacro}%=<<

\subsection{Known Bugs}
Currently only one bug is known: If after the last head row there is only one
additional row the bottom \cs{hline} will only be drawn if you end that last row
with \texttt{\string\\}. If you have more rows following the last head row,
it won't matter whether you end the last row with \texttt{\string\\} or not.
%=<<

\section{The \cs{MRTcline} macro}\label{sec:tab:cline}%>>=
\begin{describemacro}{MRTcline}%>>=
  [\meta{!}\oarg{color}\{\meta{*}\oarg{color}%
  \meta{<\oarg{left skip}}\meta{>\oarg{right skip}}\meta{cols}\}]
  Sets something like a \cs{cline} in the specified \meta{cols}.
  \\[\parskip]
  In the mandatory argument the only mandatory element is the affected
  \meta{cols}.
  \\[\parskip]
  The mandatory argument can include a comma separated list in which you can
  repeat every optional argument you like as many times as you like.
  Additionally you can enclose the \meta{cols} in curly braces and give another
  comma separated list there which then can only contain column specifications
  and none of the optional arguments using the optional arguments specified
  before that list. A valid column specification is a single column, or a column
  range separated by a \texttt{-}, so something like \meta{start-end}.
  \\[\parskip]
  Both \meta{color} arguments have the same effect, but the first applies to
  every specification in the list, while the second only affects the current
  list item. The \meta{color} doesn't change the color of the line, but the
  color of the optional fill arguments. It defaults to either
  \texttt{tabulargray} if used inside the scope of head rows, or \texttt{white}
  else. If you give a \meta{*} the current list item will be completely in the
  specified \meta{color}.
  \\[\parskip]
  You can introduce a small skip on the left side if you specify a \meta{<}
  which defaults to \verb|.5\tabcolsep|, with the optional \meta{left skip} you
  can customize that length. A small skip to the right can be introduced with
  \meta{>}, again of customizable width using \meta{right skip}.
  \\[\parskip]
  You should only use one \cs{MRTcline} per line and specify every column you
  want in that.
  \\[\parskip]
  If you don't give the optional \meta{!} after \cs{MRTcline}, before anything
  else something like a \cs{hline} using \meta{color} will be used to cover the
  full width of the tabular. This way you don't have to specify every column you
  want to color with \meta{color} using the \meta{*} type argument.
\end{describemacro}%=<<

I hope you got that rather cryptic description (if you can supply a better
description, message me as noted in \autoref{sec:bugs}).

Here are a few examples of usage with comparison to a correct \cs{cline} usage.
The source of each table is printed below it. The last example of \cs{MRTcline}
is not possible with the standard \cs{cline} as far as I know.
\begin{multicols}{2}%>>=
  \MRTthesisSetup{stretch tab=1}
  \noindent
\vbox{\DoAndPrint{\begin{MRTtabular}{lll}
  a & b & c\\
  \MRTcline{1-2}
  d & e & f\\
  g & h & i\\
  j & k & l\\
\end{MRTtabular}}}
\vbox{\DoAndPrint{\begin{MRTtabular}{lll}
  a & b & c\\
  \cline{1-2}
  \clineReveal
  d & e & f\\
  g & h & i\\
  j & k & l\\
\end{MRTtabular}}}
\end{multicols}%=<<
\begin{multicols}{2}%>>=
  \MRTthesisSetup{stretch tab=1}
  \noindent
\vbox{\DoAndPrint{\begin{MRTtabular}{lll}[2]
  a & b & c\\
  \MRTcline{1-2}
  d & e & f\\
  g & h & i\\
  j & k & l\\
\end{MRTtabular}}}
\vbox{\DoAndPrint{\begin{MRTtabular}{lll}[2]
  a & b & c\\
  \MRTcline{<>1-2}
  d & e & f\\
  g & h & i\\
  j & k & l\\
\end{MRTtabular}}}
  \columnbreak
\vbox{\DoAndPrint{\begin{MRTtabular}{lll}[2]
  a & b & c\\
  \cline{1-2}
  \arrayrulecolor{tablegray}
  \cline{3-3}
  \arrayrulecolor{black}
  \clineReveal
  \rowcolor{tablegray}
  d & e & f\\
  g & h & i\\
  j & k & l\\
\end{MRTtabular}}}
\end{multicols}%=<<
%=<<

\section{The \env{MRTtable} environment}\label{sec:tab:table}%>>=
The \env{MRTtable} environment is a wrapper around an \env{MRTtabular} inside of
a \env{table} environment. There might be a severe difference in the
implementation of the \opt{long} version but for the user this shouldn't be
noticeable. Most importantly the provided hooks at the start and end of each
line should work in the \opt{long} version, too, however they won't work if you
use another inner \env{tabular}-like environment via the \opt{env} keys.

\begin{describeenv}{MRTtable}[\oarg{key=value}]
  \env{MRTtable} sets its contents in an \env{MRTtabular} environment. It
  features several \meta{key}s you are encouraged to use.
  \\[\parskip]
  All available \meta{key}s are listed in \autoref{sec:tab:options:setup}. An
  example can be seen in \autoref{sec:tab:example}.
\end{describeenv}

\subsection{Known Bugs}
If you use the \opt{long} option together with \opt{caption below} and the table
only needs one page (so ends on the same page it started on), the caption will
be the one which would be placed on the continued pages, not the one actually
belonging on the first page. That's a limitation of the used method to place a
different foot on the first page which is not supported by \pkg{longtable}. The
method is taken from \url{https://tex.stackexchange.com/a/68448/117050}.
%=<<

\section{Explicit head rows}\label{sec:tab:explicit}%>>=
It is possible to mark head rows explicitly. For this the following macros are
provided:

\begin{describemacro}{headS}%>>=
  Start of the head rows. Sets a \cs{hline} above the current row except if the
  current row is the first row in a \env{MRTtabular} environment. Additionally
  the current row is coloured with \cs{rowcolor{tablegray}}.
\end{describemacro}%=<<

\begin{describemacro}{headR}%>>=
  An additional head row should be started with this macro. It sets the current
  row's colour to \texttt{tablegray}.
\end{describemacro}%=<<

\begin{describemacro}{headE}%>>=
  The end of the head rows. Should be used after the last row of the table's
  head but prior to the next row (immediately after \texttt{\string\\}).
\end{describemacro}%=<<

\begin{describemacro}{MRTtabDeclareHeadMacros}%>>=
  By default the above macros are only available inside of \env{MRTtabular} and
  in the body of \env{MRTtable}. \cs{MRTtabDeclareHeadMacros} will make them
  locally available.
\end{describemacro}%=<<
%=<<

\section{Other package macros}%>>=
\begin{describemacro}{MRTtabSetup}[\marg{key=value}]
  This is the interface to set the options listed in
  \autoref{sec:tab:options:setup} outside of \env{MRTtable}.
\end{describemacro}

\begin{describemacro}{clineReveal}
  As you can see in \autoref{sec:tab:cline} the macro \cs{clineReveal} is used.
  This is done because a \cs{cline} doesn't take up any vertical space (by
  issuing \verb|\noalign{\vskip-\arrayrulewidth}|) as opposed to a \cs{hline}.
  This is done so that multiple \cs{cline}s can be used in the same row. As a
  result the spacing is inconsistent and a \cs{cline} is overlapped by a
  following \cs{rowcolor} or \cs{cellcolor}. \cs{clineReveal} does introduce a
  vertical skip which reveals the lines (issuing
  \verb|\noalign{\vskip\arrayrulewidth}|).  It is also used by \cs{MRTcline}.
\end{describemacro}

\begin{describemacro}{MRTtabRepeatCols}
  This macro is to be used in column definitions of \env{tabular}s or
  \env{array}s and other macros and environments using these internally (e.g.
  \env{MRTtabular} and \env{MRTtable}). The effect is that the column
  definitions which follow this macro are repeated indefinitely to match the
  required columns for the tables body. E.g., \bverb|l \MRTtabRepeatCols c| does
  set the first column left aligned and every following column centred. It has
  to be preceded by at least one valid column definition.\\[\parskip]
  \dangerleft
  It is known that the used approach doesn't work with \env{longtable} and as
  such, it also doesn't work if you use it inside of \env{MRTtable} if it is
  using the \opt{long} option.
\end{describemacro}
%=<<

\section{Options}\label{sec:tab:options}%>>=
The package only features one load time option, which is \texttt{longtable}. If
it is specified the \pkg{longtable} package is loaded and some more options of
\env{MRTtable} become available which are focused around the usage of
\env{longtable} inside of \env{MRTtable}.

\subsection{Setup Options}\label{sec:tab:options:setup}
The following options are available for \cs{MRTtabSetup} and \env{MRTtable}.

\begin{describeopt}{align}[\meta{align}]%>>=
  If \opt{no float} has been used, a \env{minipage} is used around the
  \env{MRTtable}. With the \opt{align} option you can specify the vertical
  alignment of that \env{minipage}.
\end{describeopt}%=<<
\begin{describeopt}{caption above,above}%>>=
  If specified the caption will be put above the \env{MRTtabular} in
  \env{MRTtable}. If \cs{KOMAoptions} is available the KOMA option
  \opt{captions=tableheading} is used.
\end{describeopt}%=<<
\begin{describeopt}{caption below,below}%>>=
  If specified the caption will be put below the \env{MRTtabular} in
  \env{MRTtable}. If \cs{KOMAoptions} is available the KOMA option
  \opt{captions=tablesignature} is used.
\end{describeopt}%=<<
\begin{describeopt}{bare}[\meta{bool}]%>>=
  If set to true the potential caption and the tabular like environment in
  \env{MRTtable} are neither surrounded by a \env{minipage} nor a \env{figure}.
  Only a \cs{centering} is issued.
\end{describeopt}%=<<
\begin{describeopt}{BoL}[\meta{content}]%>>=
  Sets the \env{MRTtabular} Begin-of-Line hook using \cs{MRTtabAddtoBoLHook}
\end{describeopt}%=<<
\begin{describeopt}{EoL}[\meta{content}]%>>=
  Sets the \env{MRTtabular} End-of-Line hook using \cs{MRTtabAddtoEoLHook}
\end{describeopt}%=<<
\begin{describeopt}{caption, cap}[\meta{caption}]%>>=
  Specifies the content of the caption in an \env{MRTtable}. If it is blank, no
  caption will be used.
\end{describeopt}%=<<
\begin{describeopt}{cline version}[\meta{choice}]%>>=
  set the behaviour of \cs{MRTcline}. Choices are |1| and |2|. |2| is the
  behaviour currently described in \autoref{sec:tab:cline}. If you specify |1|
  the behaviour of the optional \meta{!} of \cs{MRTcline} is reversed.
\end{describeopt}%=<<
\begin{describeopt}{columns, col}[\meta{preamble}]%>>=
  Specifies the \env{MRTtabular} preamble (the column specifications). Defaults
  to first column \texttt{l}, others \texttt{c}.
\end{describeopt}%=<<
\begin{describeopt}{env}[\meta{name}]%>>=
  Uses the tabular like environment \meta{name} instead of \env{MRTtabular}. If
  an empty argument is provided, no inner environment will be used. This is
  useful if you want to use an environment that grabs its contents and has to be
  explicitly used, e.g. \env{tabularx} can only be used like this. If you
  specify \bverb|MRTarray| as the argument, the math mode will automatically be
  set.
\end{describeopt}%=<<
\begin{describeopt}{env begin}[\meta{begin}]%>>=
  Uses \meta{begin} as the start of the tabular like environment. This way you
  can specify some options. Note that any outer braces are stripped. If you want
  to use an environment you have to include \cs{begin} in the argument. Note
  that if the argument you provide is not empty, the column specification as
  defined with \opt{columns} is inserted in braces after \meta{begin}.
\end{describeopt}%=<<
\begin{describeopt}{env end}[\meta{end}]%>>=
  Uses \meta{end} as the end of the tabular like environment. This way you can
  specify some options. Note that any outer braces are stripped. If you want to
  use an environment you have to include \cs{end} in the argument.
\end{describeopt}%=<<
\begin{describeopt}{float}[\meta{bool}]%>>=
  If set true (the default and initial value) the \env{MRTtable} floats.
\end{describeopt}%=<<
\begin{describeopt}{head rows, head}[\meta{num}]%>>=
  The number of rows which should be formatted as head rows as in
  \env{MRTtabular}. In each \env{MRTtable} it is initially \texttt{1} -- this
  differs from the behaviour of a stand alone \env{MRTtabular} which defaults to
  \texttt{0} rows.
\end{describeopt}%=<<
\begin{describeopt}{in text sep}[\meta{skip}]%>>=
  This controls the vertical space around a non-floating \env{MRTtable}. It is
  initially set to \cs{intextsep}. If it is equal to 0pt the \cs{vskip} is not
  issued.
\end{describeopt}%=<<
\begin{describeopt}{label}[\meta{label}]%>>=
  If \opt{caption} is used the \env{MRTtable} will get the specified
  \meta{label}.
\end{describeopt}%=<<
\begin{describeopt}{no float}[\meta{bool}]%>>=
  The opposite of \opt{float}. If set true the \env{MRTtable} will not float
  which is the default (but not initial) value.
\end{describeopt}%=<<
\begin{describeopt}{no inner env}%>>=
  Same result as \opt{env~begin=\{\},env~end=\{\}}, so no tabular like
  environment is used at all.
\end{describeopt}%=<<
\begin{describeopt}{pos}[\meta{pos}]%>>=
  The placement of a floating \env{MRTtable}. Initially \texttt{tbp}.
\end{describeopt}%=<<
\begin{describeopt}{post tab, post}[\meta{content}]%>>=
  A hook which is executed right after the \cs{end} of the inner
  \env{MRTtabular}.
\end{describeopt}%=<<
\begin{describeopt}{pre tab, pre}[\meta{content}]%>>=
  A hook which is executed right before the \cs{begin} of the inner
  \env{MRTtabular}.
\end{describeopt}%=<<
\begin{describeopt}{short caption, short cap, scap}%>>=
  If caption and this option are used the list of tables will get this short
  caption instead of the caption.
\end{describeopt}%=<<
\begin{describeopt}{stretch tabular, stretch tab, stretch}[\meta{float}]%>>=
  Sets the stretch in \env{MRTtabular} to the specified \meta{float} using
  \cs{setstretch}.
\end{describeopt}%=<<
\begin{describeopt}{stretch caption, stretch cap, cstretch}[\meta{float}]%>>=
  Sets the stretch in the caption using \cs{setkomafont} and \cs{setstretch}.
  Doesn't work if KOMA script is not used but issues a warning in that case.
\end{describeopt}%=<<
\begin{describeopt}{striped}[\meta{bool}]%>>=
  If set to true the inner \env{MRTtabular} will be striped with \opt{stripe
  color 1} and \opt{stripe color 2}, beginning in line \opt{stripe start}. It
  uses \cs{rowcolors} internally.
\end{describeopt}%=<<
\begin{describeopt}{stripe color 1, stripe 1, scolor 1, scolor1}%>>=
  [\meta{color}]
  Defines the \meta{color} of the first color argument of \cs{rowcolors} if
  \opt{striped} is true. Initially set to \texttt{tablegray!50}.
\end{describeopt}%=<<
\begin{describeopt}{stripe color 2, stripe 2, scolor 2, scolor2}%>>=
  [\meta{color}]
  Defines the \meta{color} of the second color argument of \cs{rowcolors} if
  \opt{striped} is true. Initially set to \texttt{white}.
\end{describeopt}%=<<
\begin{describeopt}{stripe invert, sinvert}%>>=
  Exchanges the current values of \opt{stripe color 1} and \opt{stripe color 2}.
\end{describeopt}%=<<
\begin{describeopt}{stripe start, sstart}[\meta{row}]%>>=
  Defines the starting row of a potentially striped \env{MRTtabular}. Initially
  set to \texttt{2}.
\end{describeopt}%=<<

\subsubsection{\pkg{longtable} related options}
The following options are only available if the \opt{longtable} option was used
during package load time.

\begin{describeopt}{long}[\meta{bool}]%>>=
  If set true the \env{MRTtable} uses \env{longtable} internally. It doesn't
  float and gets page breakable. You should specify the \opt{columns} of
  \env{MRTtable} manually as the automatic detection might fail terribly in
  conjunction with \env{longtable}.
\end{describeopt}%=<<
\begin{describeopt}{continue caption, cont cap, ccap}[\meta{caption}]%>>=
  If specified following pages use this \meta{caption} instead of the \opt{short
  caption} or the normal \opt{caption}.
\end{describeopt}%=<<
\begin{describeopt}{continue with caption, cont with cap, cont w cap}%>>=
  [\meta{bool}]
  If set true, the following pages use the \opt{caption} and not the \opt{short
  caption} or \opt{continue caption}. Defaults to true and initially is set to
  false.
\end{describeopt}%=<<
\begin{describeopt}{continue text, cont text, ctext}[\meta{text}]%>>=
  The caption on following pages will be appended by this \meta{text}, this is
  true regardless of wether \opt{caption}, \opt{short caption} or \opt{continue
  caption} is used. Initially this is set to \bverb|(\emph{Fortsetzung})| (if
  \cls{MRTthesis} with the English language is used, this will be set to
  \bverb|(\emph{continued})|).
\end{describeopt}%=<<
%=<<

\section{Example}\label{sec:tab:example}
\autoref{tab:tab:example} shows an example usage of the \env{MRTtable}
environment. The code to produce it is shown below. The \opt{bare} option is
used since I placed it manually inside of a \env{minipage} right of the verbatim
listing.

\noindent
\begin{minipage}[]{.5\textwidth}
\begin{verbatim}
\begin{MRTtable}
  [
    cap=Boring Table,
    label=tab:tab:example,
    bare
  ]
  This & is & the & boring & head \\
  This & is & the & first  & line \\
  This & is & the & second & line \\
  This & is & the & third  & line \\
  This & is & the & fifth  & line \\
  This & is & the & sixth  & line \\
\end{MRTtable}
\end{verbatim}
\end{minipage}%
\begin{minipage}[]{.5\textwidth}
  \centering
  \begin{MRTtable}
    [
      cap=Boring Table,
      label=tab:tab:example,
      bare
    ]
    This & is & the & boring & head \\
    This & is & the & first  & line \\
    This & is & the & second & line \\
    This & is & the & third  & line \\
    This & is & the & fifth  & line \\
    This & is & the & sixth  & line \\
  \end{MRTtable}
\end{minipage}

\section{Dependencies}%>>=
The package requires the following packages and their dependencies:
\vspace*{-\multicolsep}%
\begin{multicols}{3}
  \begin{itemize}
    \item \pkg{expl3}
    \item \pkg{xparse}
    \item \pkg{array}
    \item \pkg{setspace}
    \item \pkg{xcolor} with option \opt{table}
    \item potentially \pkg{longtable}
  \end{itemize}
\end{multicols}
\vspace*{-\multicolsep}%
%=<<

% vim: ft=tex fdm=marker fmr=>>=,=<< sw=2 ts=2 tw=80
