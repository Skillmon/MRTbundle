\chapter{The \pkg{MRTtab} package}
The \pkg{MRTtab} provides means to typeset tables in a style similar to the ones
in the scripts of the MRT. This includes:
\begin{itemize}
  \item delimited by horizontal rules on both ends
  \item head rows are light grey and delimited by horizontal rules
  \item all horizontal rules have the same thickness
  \item no vertical rules (though not enforced)
\end{itemize}

The package provides an environment similar to \env{tabular}
(\autoref{sec:tab:tabular}), an enhanced version of \cs{cline}
(\autoref{sec:tab:cline}), and an environment to typeset displayed tables with
many options available (\autoref{sec:tab:table}).

\section{The \env{MRTtabular} environment}\label{sec:tab:tabular}%>>>
The \env{MRTtabular} environment calls a patched \env{tabular} environment. The
following differences exist:
\begin{itemize}
  \item a hook is provided at the beginning and the end of each line 
  \item above and below of it a \cs{hline} is placed
  \item it has an additional optional argument specifying the number of rows to
    be formatted as head rows.
  \item you can access the current row number
\end{itemize}
An ordinary description as done with other environments in this documentation:
\begin{describeenv}{MRTtabular}%>>>
  [\oarg{valign}\oarg{head rows}\marg{preamble}]
  The first optional argument as well as the mandatory argument match the ones
  of a regular \env{tabular} environment. \meta{head rows} specifies how many
  rows at the beginning of the environment should be formatted as head rows. If
  \meta{head rows} is not specified, no head row will be formatted. No
  further markup is required for this formatting to take place. You should end
  your rows only with \texttt{\string\\} to make the hook mechanism work (on
  which the head row markup relies).
\end{describeenv}%<<<

\begin{describemacro}{head}[\marg{num}]%>>>
  Additionally to the optional argument of \env{MRTtabular} to set the first $n$
  rows as head rows, you can use \cs{head} to set the next \meta{num} rows as
  head rows. This does not only work at the beginning of the environment but
  anywhere you want. Alternatively you can use the macros described in
  \autoref{sec:tab:explicit}.
\end{describemacro}%<<<

\begin{describemacro}{MRTtabAddtoBoLHook}[\marg{content}]%>>>
  You can add \meta{content} to the Begin-of-Line hook with this macro. Bear in
  mind that the \meta{content} should be fully expandable and not produce any
  text, if you want to use stuff like \cs{multicolumn}, \cs{rowcolor}, or
  \cs{cline} at the beginning of the line -- as this hook will be executed prior
  to that and \cs{noalign} and \cs{omit} won't work in that case. If you need
  something unexpandable you can enclose it in \cs{noalign}. The addition is
  made locally.
\end{describemacro}%<<<

\begin{describemacro}{MRTtabClearBoLHook}%>>>
  Clears the Begin-of-Line hook locally.
\end{describemacro}%<<<

\begin{describemacro}{MRTtabAddtoEoLHook}[\marg{content}]%>>>
  You can also add \meta{content} to the End-of-Line hook. Here it should not
  matter whether the contents are expandable or not, as it is impossible that
  something follows in the same row which can't follow something unexpandable.
  The addition is made locally.
\end{describemacro}%<<<

\begin{describemacro}{MRTtabClearEoLHook}%>>>
  Clears the End-of-Line hook locally.
\end{describemacro}%<<<

\begin{describemacro}{MRTtabCurrentRow}%>>>
  Returns the current row in an \env{MRTtabular} expandably.
\end{describemacro}%<<<

\subsection{Known Bugs}
Currently only one bug is known: If after the last head row there is only one
additional row the bottom \cs{hline} will only be drawn if you end that last row
with \texttt{\string\\}. If you have more rows following the last head row,
it won't matter whether you end the last row with \texttt{\string\\} or not.
%<<<

\section{The \cs{MRTcline} macro}\label{sec:tab:cline}%>>>
\begin{describemacro}{MRTcline}%>>>
  [\oarg{color}\{\meta{*}\oarg{color}%
  \meta{<\oarg{left skip}}\meta{>\oarg{right skip}}\meta{cols}\}]
  Sets something like a \cs{cline} in the specified \meta{cols}.

  In the mandatory argument the only mandatory element is the affected
  \meta{cols}.
  
  The mandatory argument can include a comma separated list in which you can
  repeat every optional argument you like as many times as you like.
  Additionally you can enclose the \meta{cols} in curly braces and give another
  comma separated list there which then can only contain column specifications
  and none of the optional arguments using the optional arguments specified
  before that list. A valid column specification is a single column, or a column
  range separated by a \texttt{-}, so something like \meta{start-end}.

  Both \meta{color} arguments have the same effect, but the first applies to
  every specification in the list, while the second only affects the current
  list item. The \meta{color} doesn't change the color of the line, but the
  color of the optional fill arguments. It defaults to either
  \texttt{tabulargray} if used inside the scope of head rows, or \texttt{white}
  else. If you give a \meta{*} the current list item will be completely in the
  specified \meta{color}.

  You can introduce a small skip on the left side if you specify a \meta{<}
  which defaults to \verb|.5\tabcolsep|, with the optional \meta{left skip} you
  can customize that length. A small skip to the right can be introduced with
  \meta{>}, again of customizable width using \meta{right skip}.

  You should only use one \cs{MRTcline} per line and specify every column you
  want in that.
\end{describemacro}%<<<

I hope you got that rather cryptic description (if you can supply a better
description, message me as noted in \autoref{sec:bugs}).

Here are a few examples of usage with comparison to a correct \cs{cline} usage.
The source of each table is printed below it. The last example of \cs{MRTcline}
is not possible with the standard \cs{cline} as far as I know.
\begin{multicols}{2}%>>>
  \MRTthesisSetup{stretch tab=1}
  \noindent
  Using \cs{MRTcline}\\[1ex]
  \begin{MRTtabular}{lll}
    a & b & c\\
    \MRTcline{1-2}
    d & e & f\\
    g & h & i\\
    j & k & l\\
  \end{MRTtabular}
  \begin{verbatim}
\begin{MRTtabular}{lll}
  a & b & c\\
  \MRTcline{1-2}
  d & e & f\\
  g & h & i\\
  j & k & l\\
\end{MRTtabular}
  \end{verbatim}
  \columnbreak
  Using \cs{cline}\\[1ex]
  \begin{MRTtabular}{lll}
    a & b & c\\
    \cline{1-2}
    \clineReveal
    d & e & f\\
    g & h & i\\
    j & k & l\\
  \end{MRTtabular}
  \begin{verbatim}
\begin{MRTtabular}{lll}
  a & b & c\\
  \cline{1-2}
  \clineReveal
  d & e & f\\
  g & h & i\\
  j & k & l\\
\end{MRTtabular}
  \end{verbatim}
\end{multicols}%<<<
\begin{multicols}{2}%>>>
  \MRTthesisSetup{stretch tab=1}
  \noindent
  \begin{MRTtabular}[][2]{lll}
    a & b & c\\
    \MRTcline{1-2,*3}
    d & e & f\\
    g & h & i\\
    j & k & l\\
  \end{MRTtabular}
  \begin{verbatim}
\begin{MRTtabular}[][2]{lll}
  a & b & c\\
  \MRTcline{1-2,*3}
  d & e & f\\
  g & h & i\\
  j & k & l\\
\end{MRTtabular}
  \end{verbatim}
  \begin{MRTtabular}[][2]{lll}
    a & b & c\\
    \MRTcline{<>1-2,*3}
    d & e & f\\
    g & h & i\\
    j & k & l\\
  \end{MRTtabular}
  \begin{verbatim}
\begin{MRTtabular}[][2]{lll}
  a & b & c\\
  \MRTcline{<>1-2,*3}
  d & e & f\\
  g & h & i\\
  j & k & l\\
\end{MRTtabular}
  \end{verbatim}
  \columnbreak
  \begin{MRTtabular}[][2]{lll}
    a & b & c\\
    \cline{1-2}
    \arrayrulecolor{tablegray}
    \cline{3-3}
    \arrayrulecolor{black}
    \clineReveal
    \rowcolor{tablegray}
    d & e & f\\
    g & h & i\\
    j & k & l\\
  \end{MRTtabular}
  \begin{verbatim}
\begin{MRTtabular}[][2]{lll}
  a & b & c\\
  \cline{1-2}
  \arrayrulecolor{tablegray}
  \cline{3-3}
  \arrayrulecolor{black}
  \clineReveal
  \rowcolor{tablegray}
  d & e & f\\
  g & h & i\\
  j & k & l\\
\end{MRTtabular}
  \end{verbatim}
\end{multicols}%<<<

\begin{describemacro}{clineReveal}
  As you can see above the macro \cs{clineReveal} is used. This is done because
  a \cs{cline} doesn't take up any vertical space (by issuing
  \verb|\noalign{\vskip-\arrayrulewidth}|) as opposed to a \cs{hline}. This is
  done so that multiple \cs{cline}s can be used in the same row. As a result the
  spacing is inconsistent and a \cs{cline} is overlapped by a following
  \cs{rowcolor} or \cs{cellcolor}. \cs{clineReveal} does introduce a vertical
  skip which reveals the lines (issuing \verb|\noalign{\vskip\arrayrulewidth}|).
  It is also used by \cs{MRTcline}.
\end{describemacro}
%<<<

\section{The \env{MRTtable} environment}\label{sec:tab:table}%>>>
%<<<

\section{Explicit head rows}\label{sec:tab:explicit}%>>>
It is possible to mark head rows explicitly. For this the following macros are
provided:

\begin{describemacro}{headS}%>>>
  Start of the head rows. Sets a \cs{hline} above the current row except if the
  current row is the first row in a \env{MRTtabular} environment. Additionally
  the current row is coloured with \cs{rowcolor{tablegray}}.
\end{describemacro}%<<<

\begin{describemacro}{headR}%>>>
  An additional head row should be started with this macro. It sets the current
  row's colour to \texttt{tablegray}.
\end{describemacro}%<<<

\begin{describemacro}{headE}%>>>
  The end of the head rows. Should be used after the last row of the table's
  head but prior to the next row (immediately after \texttt{\string\\}).
\end{describemacro}%<<<

\begin{describemacro}{MRTtabDeclareHeadMacros}%>>>
  By default the above macros are only available inside of \env{MRTtabular} (and
  therefore also in the body of \env{MRTtable}). You can make them locally
  available with \cs{MRTtabDeclareHead}.
\end{describemacro}%<<<
%<<<
