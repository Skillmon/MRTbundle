\chapter{Introduction}
This bundle provides two \LaTeX\ classes, one for theses and one for
presentations, which both aim to match the corresponding MS Office templates of
the institute Lehrstuhl für Mess- und Regeltechnik (MRT) of the University of
Bayreuth, hence the name. Along the two major classes \cls{MRTthesis} and
\cls{MRTbeam} there are minor auxiliary packages contained in this distribution.

This bundle makes no claim to be complete, comprehensive, or correct. For
formatting errors I don't take any responsibility. Each author takes full
liability for his work and its formatting.

You're allowed to share this work with fellow students working at the MRT,
though official distribution channels might be better suited as they assure up
to date versions.

I'd feel guilty distributing this bundle without saying the following: I'm not
responsible for the overall look of this. I tried to match the Word template of
the institution where possible and as a result, this is non-optimal typography,
in my humble opinion.

Of course this documentation is created with one of the provided classes, namely
\cls{MRTthesis}, in use.

If you're not yet familiar with \LaTeX\ you should stop reading at this point
(meaning the end of this paragraph) and either read a \emph{good} and
\emph{up-to-date} introduction to \LaTeX\ and afterwards read on or use MS Word
for your thesis. Personally I think the time reading an introduction in order to
use \LaTeX\ is well spend, but there certainly are different opinions on that --
unfortunately opinions are prone to be biased, mine is no exception. A viable
introduction is lshort which is available in several languages at the following
link: \url{https://www.ctan.org/pkg/lshort}

\section{Feature Requests and Bug Reports}\label{sec:bugs}
You can request features or report bugs if you find some via email:
\href{mailto:mrt_depp@yahoo.de?subject=MRTbundle -- bug report}
  {mrt\_depp@yahoo.de}.
Please use a descriptive subject containing ``MRTbundle'' (e.g. ``MRTbundle --
bug report'').

\section{Individual Versions}
\docIndividualVersions
