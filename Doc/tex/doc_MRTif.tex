\chapter{The \pkg{MRTif} package}
The \pkg{MRTif} package provides a number of expandable tests. In the following
macros \tfarg\ is used to specify that the macros exist with the endings
\texttt{T}, \texttt{F}, and \texttt{TF}. The \texttt{T} ending stands for a
\meta{true} branch, \texttt{F} for the \meta{false} branch.

If a macro name contains a \texttt{G} prior to \tfarg, it strips any outermost
groups prior to the test using \cs{MRTifGroupTF}. An \texttt{N} denotes that the
first token in the argument is expanded once prior to any test. If a macro which
takes two arguments ends with \texttt{NN} prior to the \tfarg\ in both arguments
the first token is expanded once, \texttt{Nn} and \texttt{nN} mean that only for
the first and second argument, respectively, an expansion is made.

\pkg{MRTif} uses a special marker in some of its tests which expands to the
undefined control sequence
\begin{center}
  \ttfamily
  \detokenize\expandafter\expandafter\expandafter
    {\csname endMRTif@argument\endcsname}%
\end{center}
If you ever see this in your log or console output, please contact me as stated
in \autoref{sec:bugs} and include a minimal example producing this behaviour in
your contacting. Please do the same if you get any other undefined control
sequence errors containing \texttt{MRTif} in the control sequence's name.

\section{Macros}
\begin{describemacro}{MRTifCreateBranchingIfs}
  [\marg{base}\marg{branches}\marg{args}\marg{if}]
  This macro creates different branching if-tests from a \TeX-like \meta{if}.
  \meta{base} is the base name of the new macros (e.\,g.\@ |MRTifEmpty|),
  \meta{branches} is a comma separated list of different name extensions and
  branches (a detailed description follows), \meta{args} is the number of
  arguments the \TeX-like \meta{if} will take and \meta{if} should be a
  \TeX-like \meta{if}, meaning an \cs{if...} test which would need a \cs{fi}.\\
  The \meta{branches} can be an arbitrary number (though more than four don't
  make sense) of \texttt{\meta{key}=\meta{value}} pairs separated by commas, you
  should make sure that they don't contain any spaces as those aren't stripped.
  \meta{key} can be anything and will append the \meta{base} name to form the
  individual macro name while \meta{value} should be one of the values shown in
  \autoref{tab:if:cbi}. Each resulting macro name built from \meta{base} and
  \meta{key} must be undefined else it will be skipped (the other names might
  still become defined).\\
  For example the \cs{MRTifNumToken...} macros are defined
  using:\\[\smallskipamount]
  \begin{minipage}[c]
    {%
      \dimexpr
      \textwidth + \marginparwidth -
      \csname l_docMRTbundle_dscmac_widest_dim\endcsname
      \relax
    }
\begin{verbatim}
\MRTifCreateBranchingIfs{MRTifNumToken}{TF=ab,T=yn,F=ny}{2}
  {\ifnum#1=\MRTtllength{#2} }
\end{verbatim}
  \end{minipage}\strut\\[\smallskipamount]
  The space after |{#2}| is intended there to end the number parsing of |\ifnum|
  and does no harm (just in case you wondered). With |TF=ab| we define that the
  macro ending with |TF| should execute the first branch if the |\ifnum| yields
  |true| and else execute the second branch. Similarly |T=yn| defines the |T|
  macro to execute the next branch if |\ifnum| is true and else gobbles it.
  \begin{MRTtable}
    [
      ,col=>{\ttfamily}cccc
      ,cap=
        {
          Possible \meta{value}s in \cs{MRTifCreateBranchingIfs}'s
          \meta{branches} argument, how many branches they'll create and which
          branch will be executed.
        }
      ,scap=
        {
          Possible \meta{value}s in \cs{MRTifCreateBranchingIfs}'s
          \meta{branches} argument.
        }
      ,label=tab:if:cbi
    ]
    \multicolumn{1}{c}{\meta{value}} & number of branches & if true & if false\\
    ab & 2 & first  & second \\
    ba & 2 & second & first  \\
    yn & 1 & first  & gobble \\
    ny & 1 & gobble & first  \\
  \end{MRTtable}
\end{describemacro}

\begin{describemacroTF}{MRTifMathMode}
  Tests if you're currently in math mode. This is just a wrapper around the
  primitive \cs{ifmmode}.
\end{describemacroTF}

\begin{describemacroTF}[G,N,GN]{MRTifEmpty}[\marg{arg}]
  Tests if \meta{arg} is completely empty.
\end{describemacroTF}
\dscremaininglines{2}

\begin{describemacroTF}[G,N,GN]{MRTifBlank}[\marg{arg}]
  Tests if \meta{arg} is completely empty or contains only spaces.
\end{describemacroTF}
\dscremaininglines{2}

\begin{describemacroTF}[N]{MRTifGroup}[\marg{arg}]
  Tests if \meta{arg} is a single group no matter what the contents of that
  group are. It ignores spaces around the group.
\end{describemacroTF}

\begin{describemacroTF}[N]{MRTifGroupNoSpaces}[\marg{arg}]
  Tests if \meta{arg} is a single group no matter what the contents of that
  group are. It doesn't ignore spaces around the group.
\end{describemacroTF}

\begin{describemacroTF}[NN,Nn,nN,G,GNN,GNn,GnN]{MRTifStringsMatch}%
  [\marg{string1}\marg{string2}]
  Tests if \meta{string1} and \meta{string2} match, the strings are
  \cs{detokenize}d prior to the comparison.
\end{describemacroTF}
\dscremaininglines{4}

\begin{describemacroTF}[G]{MRTifStringsMatchXX}[\marg{string1}\marg{string2}]
  Tests if \meta{string1} and \meta{string2} match, the strings are fully
  expanded.
\end{describemacroTF}

\begin{describemacroTF}[G,N,GN]{MRTifOneToken}[\marg{arg}]
  Tests if \meta{arg} is only a single token or group, leading spaces are
  ignored.
\end{describemacroTF}
\dscremaininglines{2}

\begin{describemacroTF}[N]{MRTifOneTokenNoGroup}[\marg{arg}]
  Tests if \meta{arg} is only a single token. A single group is also
  \meta{false}. A |G| version is not supplied for obvious reasons.
\end{describemacroTF}

\begin{describemacroTF}[G,N,GN]{MRTifTwoToken}[\marg{arg}]
  Tests if \meta{arg} is exactly two tokens or groups, leading spaces and spaces
  between the first two tokens are ignored.
\end{describemacroTF}
\dscremaininglines{1}

\begin{describemacroTF}[G,N,GN]{MRTifNumToken}[\marg{num}\marg{arg}]
  Tests if \meta{arg} is exactly \meta{num} tokens long. It uses
  \cs{MRTtllength} internally. Compared to \cs{MRTifOneToken} and
  \cs{MRTifTwoToken} this macro takes longer and the longer the tested
  \meta{arg} the longer it takes. The |G| and |N| variants only work on
  \meta{arg}, \meta{num} will not be changed.
\end{describemacroTF}

\begin{describemacroTF}[G,N,GN]{MRTifNumTokenS}[\marg{num}\marg{arg}]
  Like \cs{MRTifNumTokenTF} but this one uses \cs{MRTtllengthS} and therefore
  captures spaces.
\end{describemacroTF}

\begin{describemacroTF}[G,N,GN]{MRTifLetter}[\marg{arg}]
  Tests if \meta{arg} is a letter, meaning of category code 11.
\end{describemacroTF}
\dscremaininglines{2}

\begin{describemacroTF}[NN,Nn,nN,G,GNN,GNn,GnN]{MRTifTokensMatch}%
  [\marg{arg1}\marg{arg2}]
  Tests if \meta{arg1} and \meta{arg2} are single tokens and if so compares
  them whether both tokens match. The variants without \texttt{G} test if one of
  the arguments is contained in a group. If that's the case the \meta{false}
  branch is executed.
\end{describemacroTF}
\dscremaininglines{4}

\begin{describemacroTF}[G,N,GN]{MRTifDigit}[\marg{arg}]
  Tests if \meta{arg} is a single token and a digit.
\end{describemacroTF}
\dscremaininglines{2}

\begin{describemacroTF}[G,N,GN]{MRTifNumber}[\marg{arg}]
  Tests if \meta{arg} is a number, meaning it consists out of an optional
  \texttt{+} or \texttt{-} sign and digits.
\end{describemacroTF}
\dscremaininglines{1}

\begin{describemacroTF}[G,N,GN]{MRTifNumberNoSign}[\marg{arg}]
  Same as \cs{MRTifNumberTF} but also returns \meta{false} for a leading sign.
\end{describemacroTF}
\dscremaininglines{2}

\begin{describemacroTF}[G,N,GN]{MRTifFloat}[\marg{arg}]
  Tests if \meta{arg} is a float, meaning it consists out of an optional
  \texttt{+} or \texttt{-} sign, optional digits, an optional decimal marker
  (\texttt{.}) and digits (which are again optional if there were digits prior
  to a decimal marker).
\end{describemacroTF}

\begin{describemacroTF}[G,N,GN]{MRTifFloatNoSign}[\marg{arg}]
  Same as \cs{MRTifFloatTF} but also returns \meta{false} for a leading sign.
\end{describemacroTF}
\dscremaininglines{2}

\begin{describemacroTF}[G,N,GN]{MRTifContainsGroup}[\marg{arg}]
  Tests if \meta{arg} contains any braced groups.
\end{describemacroTF}
\dscremaininglines{2}

\begin{describemacroTF}[G,N,GN]{MRTifContainsSpace}[\marg{arg}]
  Tests if \meta{arg} contains spaces which are not enclosed by inner groups.
\end{describemacroTF}
\dscremaininglines{2}

\begin{describemacroTF}[NN,Nn,nN,G,GNN,GNn,GnN]{MRTifTokenIn}
  [\marg{token}\marg{token list}]
  Tests whether \meta{token list} contains \meta{token}. The group variant only
  strips outer groups for \meta{token list}. Any inner group in \meta{token
  list} is skipped (so one can hide tokens from this search). The test is slower
  than non-expandable alternatives doing the same because it scans \meta{token
  list} recursively. \meta{token} should be a single token, if it's empty the
  test is true, if it is a space \cs{MRTifContainsSpaceTF} is used, if it is
  more than a single token the test is false, also a single group as an argument
  yields false.
\end{describemacroTF}

\begin{describemacroTF}[NN,Nn,nN]{MRTifTokenInDeep}
  [\marg{token}\marg{token list}]
  This does the same as \cs{MRTifTokenInTF}, except that there is no hiding!
  Therefore no |G| version is provided.
\end{describemacroTF}

\begin{describemacro}{MRTtllength,MRTtllengthN}[\marg{arg}]
  Expands to the number of tokens or groups inside of \meta{arg}. Unprotected
  spaces are ignored. The ordinary version needs two expansions while the |N|
  version needs four. A group is counted as one Token.
\end{describemacro}

\begin{describemacro}{MRTtllengthS,MRTtllengthSN}[\marg{arg}]
  Like \cs{MRTtllength} but this version counts spaces, too.
\end{describemacro}

\begin{describemacro}{MRTifFexp,MRTifFexpI,MRTifFexpII}%
  [\marg{MRTif test}\meta{branches}]
  These macros take an arbitrary expandable test and expand it in exactly two
  steps of expansion. \cs{MRTifFexp} can be applied to any test, while
  \cs{MRTifFexpI} is meant to be used for tests having only one -- a true or a
  false -- branch and \cs{MRTifFexpII} is meant to be used for tests having two
  branches. \meta{MRTif test} doesn't necessarily have to be a test provided by
  \pkg{MRTif} but can be any fully expandable test. Inside of \meta{MRTif test}
  all the arguments necessary for the test should be contained but not the true
  or false branch. An example:\\[\smallskipamount]
  \mbox{}\hfill\begin{tabular}{@{}l@{}}
    \verb|\MRTifFexp{\MRTifEmptyF{abc}}{false}|          \\
    \verb|\MRTifFexp{\MRTifEmptyTF{abc}}{true}{false}|   \\
    \verb|\MRTifFexpI{\MRTifEmptyF{abc}}{false}|         \\
    \verb|\MRTifFexpII{\MRTifEmptyTF{abc}}{true}{false}| \\
  \end{tabular}\hfill\mbox{}%
  \\[\smallskipamount]
  all expand to |false| after exactly two steps of expansion. \cs{MRTifFexpI}
  and \cs{MRTifFexpII} are more than thrice as fast as \cs{MRTifFexp} and of
  course each test takes longer with these added than without.%
  \footnote
    {%
      Benchmarking done with \hologo{pdfTeX}, version 3.14159265-2.6-1.40.19
      (\TeX\ Live 2018), on an Intel\textregistered\ Core\texttrademark\
      \mbox{i5-2540M} with \cs{MRTifEmptyT} and \cs{MRTifEmptyTF} utilizing the
      \pkg{l3benchmark} package. To give some numbers: 2~\cs{MRTifFexpI} and
      2~\cs{MRTifFexpII} added roughly \SI{1.2}{\micro\second} to the compile
      time of \SI{1.8}{\micro\second} for 2~\cs{MRTifEmptyT} and
      2~\cs{MRTifEmptyTF} tests, each once empty and once not with empty
      branches, while 4~\cs{MRTifFexp} added \SI{4.2}{\micro\second}.%
    }
  The advantage is the control
  over the needed expansion steps.
\end{describemacro}

\section{Dependencies}
\pkg{MRTif} loads the \pkg{pdftexcmds} package to make the \hologo{pdfTeX}
primitive \cs{pdfstrcmp} available as \cs{pdf@strcmp} for \hologo{LuaTeX}.
Additionally it uses \pkg{MRTutil}.
