\chapter{The \pkg{MRTcirc} package}
The package \pkg{MRTcirc} loads the \pkg{circuitikz} package and sets some of
its options so that the results resemble the circuits in the scripts of the MRT
better.

Additionally to those changes it adds a few options and elements to
\pkg{circuitikz}.

The version numbers of \pkg{MRTcirc} mimic the numbers of \pkg{circuitikz} to
indicate with which version it is compatible. The first version it was
compatible with was v1.2.1 released on 2020-07-06. The only guaranteed version
it works with is the one indicated by \pkg{MRTcirc}'s current version number,
but it might also work with older versions starting with v1.2.1, however a
warning will be thrown in that case.

The current compatible \pkg{circuitikz} version is
\begingroup
\ExplSyntaxOn
\cs_set:Npn \docMRTbundle_qd_show_version:n #1
  {
    %\tl_analysis_show:n { #1 }
    v
    \tl_if_in:nnTF { #1 } { - }
      { \docMRTbundle_qd_show_version_tl:w #1 \q_stop }
      { #1 }
    .
  }
\cs_generate_variant:Nn \docMRTbundle_qd_show_version:n { v }
\cs_set:Npn \docMRTbundle_qd_show_version_tl:w #1 - #2 \q_stop
  {
    \docMRTbundle_qd_show_version_aux:nn { #1 } { #2 }
  }
\cs_set:Npn \docMRTbundle_qd_show_version_aux:nn #1 #2
  {
    #1~(\pkg{MRTcirc}~release~#2~for~this~version)
  }
\docMRTbundle_qd_show_version:v { MRTcirc@version }
\ExplSyntaxOff
\endgroup

\section{Added Elements}%>>=
The elements are listed in the style of the \pkg{circuitikz} manual. They behave
just like the elements of the section from \pkg{circuitikz}'s documentation, so,
e.g., since the |european comp port| is a European logic gate, it has the same
anchors as the other European logic gates defined by \pkg{circuitikz}. Note
however that if no American or IEEE variant is listed in the following, it means
that it isn't provided, still you should be able to drop the |european| from the
name if the |european|-option of \pkg{circuitikz} is used (which it is if you're
not altering how \pkg{MRTcirc} loads \pkg{circuitikz}).

\begingroup
\def\arraystretch{1.408}%
\newcommand\ctikzman[2]{\multicolumn{2}{l}{\textrm{\textbf{#1\quad#2}}}\\}%
\settowidth{\dimen0}{\includegraphics{img/circ_comp-alone.pdf}}%
\edef\tmp{\the\dimexpr\linewidth-4\tabcolsep-\dimen0}%
\adjustboxset{valign=t}%
\begin{longtable}{rp{\tmp}}
  \ctikzman{3.24.3}{European Logic gates}
  \adjincludegraphics{img/circ_comp-alone.pdf}
    & European comparator port, type: node, fillable
      (\texttt{node[european comp port]\{\}}). Class: \texttt{logic ports}. \\
  \adjincludegraphics{img/circ_clock-alone.pdf}
    & European clock port, type: node, fillable
      (\texttt{node[european clock port]\{\}}). Class: \texttt{logic ports}. \\
  \ctikzman{3.24.4}{Path-style logic ports}
  \adjincludegraphics{img/circ_comp-alone.pdf}
    & \texttt{inline comp}: ``comp'' logic port, \texttt{type: path-style,
    fillable, nodename: comp port}. Class: \texttt{logic ports}.
\end{longtable}
\endgroup
%=<<

\section{Altered Elements}%>>=
While \pkg{circuitikz} provides many customization options, not everything can
be customized so much that it fits. For this reason \pkg{MRTcirc} alters some of
the definitions made by \pkg{circuitikz} to get greater flexibility.

\subsection{Flip-flops}%>>=
The flip-flops are changed a lot from the basic definition. You can still use
everything from the manual, but additionally two new keys were added which you
can use in |flipflop def| (or in |\ctikzset{multipoles/flipflop/|\meta{key}|}|):

\begin{describeopt}{clock wedge stretch}[\meta{bool}]
  If |true| the wedges are stretched so that they look pointier, just like in
  the scripts of the MRT. If |false| the default of \pkg{circuitikz} is used.
  Initially this is |true|.
\end{describeopt}
\begin{describeopt}{dashed}[\meta{bool}]
  If |true| the flip-flops get a dashed horizontal line at their vertical
  centre, just like they are depicted in the MRT scripts. Initially this is
  |true|.
\end{describeopt}

A display of the altered flip-flops:

\begingroup
\def\arraystretch{1.408}%
\newcommand\ctikzman[2]{\multicolumn{2}{l}{\textrm{\textbf{#1\quad#2}}}\\}%
\settowidth{\dimen0}{\includegraphics{img/circ_flipflop_asr-alone.pdf}}%
\edef\tmp{\the\dimexpr\linewidth-4\tabcolsep-\dimen0}%
\newcommand\flfl[3]
  {%
    \adjincludegraphics{img/circ_flipflop_#1-alone.pdf}
      & \texttt{\string\node[flipflop #2]\{\};} (#3)
      \\
  }%
\adjustboxset{valign=c}%
\begin{longtable}{cp{\tmp}}
  \flfl{asr}{aSR}{asynchronous SR-flip-flop}
  \flfl{sr}{SR}{synchronous SR-flip-flop}
  \flfl{d}{D}{synchronous D-flip-flop}
  \flfl{t}{T}{synchronous T-flip-flop}
  \flfl{jk}{JK}{synchronous JK-flip-flop}
\end{longtable}
\endgroup

In addition each of the flip-flops listed above now accepts an optional argument
such as |\node[flipflop SR={|\meta{flipflop def}|}]{};|, with
\meta{flipflop def} being any of the allowed customizations for the base
|flipflop| style of \pkg{circuitikz}. This way you can alter them more easily.

As you might see, none of the flip-flops have outputs, you can use the anchor
|.bpin 4| for the output $\overline{\mathrm{Q}}$ and |.bpin 6| for the output
$\mathrm{Q}$. This way you don't have to always specify which of the outputs is
actually used and which isn't. If you're a perfectionist and want identical
lengths you can show one of the pins, e.g., you can show the negated output by
using |flipflop |\meta{type}|={t4=\relax}| and then use the anchor |.pin 4|
instead.

An example of a small circuit using a flip-flop with a reset, a standard clock
wedge and without the dashed line:\par
\noindent
\begingroup
\settowidth{\dimen0}{\includegraphics{img/circ_flipflop_example-alone.pdf}}%
\edef\tmp{\the\linewidth-\dimen0-1cm}%
\adjincludegraphics[valign=c]{img/circ_flipflop_example-alone.pdf}\hfill
\begin{minipage}[c]{\tmp}
  \scriptsize
\begin{verbatim}
\begin{circuitikz}
  \draw
    node[flipflop T={td=R,dashed=false,clock wedge stretch=false}](T){}
    (T.pin 1) to[short,-o] ++(-.5,0)node[left]{T}
    (T.pin 2) to[short,-o] ++(-.5,0)node[left]{CLK}
    (T.bpin 4) to[short,-o] ++(.5,0)node[right]{\ctikztextnot{Q}}
    (T.bpin 6) to[short,-o] ++(.5,0)node[right]{Q}
    (T.down) to[short,-o] ++(0,-.5)node[below]{R}
    ;
\end{circuitikz}
\end{verbatim}
\end{minipage}
\endgroup
%=<<
%=<<

\section{Dependencies}%>>=
Since it builds upon and alters the behaviour of \pkg{circuitikz} the first
dependency might not be hard to guess.

\begin{multicols}{2}%>>=
  \begin{itemize}[leftmargin=10pt]
    \item \pkg{circuitikz} [european, straightvoltages]
    \item \pkg{etoolbox}
  \end{itemize}
\end{multicols}%=<<
%=<<
