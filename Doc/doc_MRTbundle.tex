\documentclass[oneside]{MRTthesis}

\usepackage{collcell}

\MRTthesisSetup{stretches=1,stretch tab=1.408}

% documentation macros >>>
\newcommand*\pkg[1]{\textrm{#1}}
\newcommand*\cls[1]{\textrm{#1}}
\newcommand*\env[1]{\texttt{#1}}
\long\def\eatspace#1 #2{#1#2}
\newcommand*\cs{\texorpdfstring{\csTeXString}{\csPDFString}}
\protected\def\csTeXString#1%>>>
  {%
    \texttt{\expandafter\eatspace\string\ \detokenize{#1}}%
  }%<<<
\def\csPDFString#1{(macro #1)}
\protected\def\meta#1{\texttt{$\langle$\textit{#1}$\rangle$}}
\newcommand*\metaEnclosed[3]{\texttt{#1}\meta{#2}\texttt{#3}}
\newcommand*\marg[1]{\metaEnclosed{\{}{#1}{\}}}
\newcommand*\oarg[1]{\metaEnclosed{[}{#1}{]}}
\newcommand*\barg[1]{\metaEnclosed{(}{#1}{)}}
\newcommand*\carg[2]{\texttt{(}\meta{#1},\meta{#2}\texttt{)}}
% <<<

\begin{document}
\tableofcontents
\mainpart
\chapter{Introduction}
I feel guilty distributing this bundle without saying the following: I'm not
responsible for the overall look of this. I tried to match the Word template of
the institution where possible and as a result, this is non-optimal typography.

Of course this documentation is created with one of the provided classes, namely
\cls{MRTthesis}, in use.

\chapter{The \cls{MRTthesis} class}
\chapter{The \cls{MRTbeam} class}
\chapter{The \pkg{MRTif} package}
\chapter{The \pkg{MRTsfacc} package}
This package is provided to remedy an issue related with sans serif maths, to be
more precise to fix the placement of \cs{mathaccentV}, which is internally used
by macros such as \cs{hat} and \cs{dot}. It is therefore loaded by both,
\cls{MRTthesis} and \cls{MRTbeam}. The \cls{beamer} class provides a fix for the
same issue which is unfortunately only working for \cls{beamer}'s default font
by fixing the font metrics.

\cls{MRTsfacc} has a different approach by patching \cs{mathaccentV} to move the
accent horizontally depending on the height of the accented character.
Furthermore it is tested whether the character is an alphabetic one by checking
the category code. If it is not an alphabetic character the shift isn't applied.

One can use \texttt{*} to enforce the shift and \texttt{!} to enforce the
omitting of that shift. Consider the following example:
\verb$\hat!{m}$ produces $\hat!{m}$, \verb$\hat*{m}$ produces $\hat*{m}$, and
\verb$\hat{m}$ produces $\hat{m}$, which is the same as \verb$\hat*{m}$ since
\texttt{m} has by default the category code of an alphabetic character.

The tokens \texttt{*} and \texttt{!} must not be enclosed by braces if you want
to specify the behaviour of \cs{mathaccentV}, so \verb$\hat{*}\hat{!}$ results
in $\hat{*}\hat{!}$.

\begin{table}
  \centering
  \setstretch{1.408}%
  \newcommand\hatex[1]{$\hat!{#1}$}%
  \newcommand\hatst[1]{$\hat*{#1}$}%
  \begin{tabular}
    {%
      *2{>{\collectcell\hatex}c<{\endcollectcell}}
      *2{>{\collectcell\hatst}c<{\endcollectcell}}
    }
    \hline
    \rowcolor{tablegray}
    \multicolumn{2}{c}{original} & \multicolumn{2}{c}{shifted} \\ 
    \hline
    a & A & a & A \\
    b & B & b & B \\
    c & C & c & C \\
    d & D & d & D \\
    e & E & e & E \\
    f & F & f & F \\
    g & G & g & G \\
    h & H & h & H \\
    i & I & i & I \\
    j & J & j & J \\
    k & K & k & K \\
    l & L & l & L \\
    m & M & m & M \\
    n & N & n & N \\
    o & O & o & O \\
    p & P & p & P \\
    q & Q & q & Q \\
    r & R & r & R \\
    s & S & s & S \\
    t & T & t & T \\
    u & U & u & U \\
    v & V & v & V \\
    w & W & w & W \\
    x & X & x & X \\
    y & Y & y & Y \\
    z & Z & z & Z \\
    \hline
  \end{tabular}
  \caption{Comparison of shifted accents against original placement}
  \label{tab:sfacc}
\end{table}

The resulting \cs{mathaccentV} macro is not expandable and therefore created
\cs{protected}. You can take a look at the results at table \ref{tab:sfacc}.

\chapter{Bug reports}
You can report bugs if you find some via email:
\href{mailto:mrt_depp@yahoo.de?subject=MRTbundle -- bug report}
  {mrt\_depp@yahoo.de}

\end{document}
