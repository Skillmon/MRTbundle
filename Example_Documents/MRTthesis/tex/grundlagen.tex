\chapter{Grundlagen}
\begingroup
\color{blue}
\section{Blabla}
\subsection{Blablabla}
Man muss ein bisschen was über Grundlagen schreiben. Aber nicht die halbe Arbeit
lang. Falls Formeln nötig sind, benutzt man am besten die Umgebung
\verb|\begin{align}|\hskip0pt\verb|...|\hskip0pt\verb|\end{align}|
bei mehreren aufeinander folgenden Formeln oder, sollte nur eine Formel gesetzt
werden, 
\verb|\begin{equation}|\hskip0pt\verb|...|\hskip0pt\verb|\end{equation}|.

Ein Beispiel für eine wichtige Formel ist
\begin{equation}
  1+1 = 2\,\text{.}
  \label{eq:1}
\end{equation}
Andere wichtige, ebenfalls durchnummerierte Formeln neben \autoref{eq:1} sind
\begin{align}
  1+2 &= 3\,\text{,}\label{eq:2}\\
  1+3 &= 4\text{ und}\label{eq:3}\\
  1+4 &= 5\,\text{.}\label{eq:4}
\end{align}

Beim Editieren von Formeln muss man eigentlich auf nichts achten, was die
Schriftart anbelangt. Wichtig ist, dass man keine leere Zeile vor oder nach
Gleichungen lässt, sollte man keinen neuen Absatz starten wollen. Vor einer
Gleichungsumgebung sollte dies nie der Fall sein.
\newpage
Symbole im Mathemodus:

\def\X#1{$#1$&$\mathrm{#1}$&\string#1}
\def\Y#1{\multicolumn{2}{l}{$#1$}&\string#1}
\newcounter{countA}%
\newcommand{\myline}[2][0]{%
    \ifnum#1>0\setcounter{countA}{#1}\fi%
    \stepcounter{countA}\Alph{countA}$\Alph{countA}$\alph{countA}$\alph{countA}$&%
    \ifnum\numexpr#2-1>\value{countA}%
    \myline{#2}%
    \else%
    \stepcounter{countA}\Alph{countA}$\Alph{countA}$\alph{countA}$\alph{countA}$%
    \setcounter{countA}{0}%
    \fi%
}
\noindent
\begin{tabular}{l?l}
    \myline{10} \\
    \myline[10]{20}\\
    \myline[20]{26}&
    $i\times i$ & $i\cdot i$ & $i\circ i$ & $i=i$ \\
    $i>i$ & $i<i$ & $i\in i$ & $i+i$ & $i-i$ & $i/i$
\end{tabular}\\
\begin{tabular}{l?l}
\X\alpha        &\X\theta       &\X o           &\X\tau         \\
\X\beta         &\X\vartheta    &\X\pi          &\X\upsilon     \\
\X\gamma        &\X\gamma       &\X\varpi       &\X\phi         \\
\X\delta        &\X\kappa       &\X\rho         &\X\varphi      \\
\X\epsilon      &\X\lambda      &\X\varrho      &\X\chi         \\
\X\varepsilon   &\X\mu          &\X\sigma       &\X\psi         \\
\X\zeta         &\X\nu          &\X\varsigma    &\X\omega       \\
\X\eta          &\X\xi                                          \\
                                                                \\
\X\Gamma        &\X\Lambda      &\X\Sigma       &\X\Psi         \\
\X\Delta        &\X\Xi          &\X\Upsilon     &\X\Omega       \\
\X\Theta        &\X\Pi          &\X\Phi                         \\
                                                                \\
\Y\arccos & \Y\cos    & \Y\csc & \Y\exp \\
\Y\arcsin & \Y\cosh   & \Y\deg & \Y\gcd \\
\Y\arctan & \Y\cot    & \Y\det & \Y\hom \\
\Y\arg    & \Y\coth   & \Y\dim & \Y\inf \\
\Y\ker    & \Y\limsup & \Y\min & \Y\sinh \\
\Y\lg     & \Y\ln     & \Y\Pr  & \Y\sup  \\
\Y\lim    & \Y\log    & \Y\sec & \Y\tan  \\
\Y\liminf & \Y\max    & \Y\sin & \Y\tanh \\
                                         \\
\Y\int    & \Y\sum  & \Y\prod  & \Y\coprod \\
\Y\oint   & \Y\iint & \Y\iiint & \Y\iiiint \\
\Y\idotsint & \Y\in & \Y\forall & \Y\exists
\end{tabular}
\endgroup
