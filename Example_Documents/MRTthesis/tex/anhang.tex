\begingroup
\color{blue}
\addchap{Anhang}
\section{Lange Rechnung}
\begin{align}%>>>
    1+1&=2\\
    1+2&=3\\
    1+3&=4\\
    1+4&=5\\
    1+5&=6\\
    1+6&=7\\
    1+7&=8\\
    1+8&=9\\
    1+9&=10\\
    1+10&=11\\
    1+11&=12\\
    1+12&=13\\
    1+13&=14\\
    1+14&=15\\
    1+15&=16\\
    1+16&=17\\
    1+17&=18\\
    1+18&=19\\
    1+19&=20\\
    1+20&=21
\end{align}%<<<

\section{Lange Tabelle}
\begin{MRTtable}
  [
    ,cap=
      {%
        Lange Tabelle im Anhang mit richtig langer Überschrift, die zumindest
        unschön klingt.%
      }
    ,scap={Lange Tabelle im Anhang}
    ,col=*{7}{c}
    ,label=tab:long
    ,striped
  ]
  Beispiel & Beispiel & Beispiel & Beispiel & Beispiel & Beispiel & Beispiel\\
  Beispiel & Beispiel & Beispiel & Beispiel & Beispiel & Beispiel & Beispiel\\
  Beispiel & Beispiel & Beispiel & Beispiel & Beispiel & Beispiel & Beispiel\\
  Beispiel & Beispiel & Beispiel & Beispiel & Beispiel & Beispiel & Beispiel\\
  Beispiel & Beispiel & Beispiel & Beispiel & Beispiel & Beispiel & Beispiel\\
  Beispiel & Beispiel & Beispiel & Beispiel & Beispiel & Beispiel & Beispiel\\
  Beispiel & Beispiel & Beispiel & Beispiel & Beispiel & Beispiel & Beispiel\\
  Beispiel & Beispiel & Beispiel & Beispiel & Beispiel & Beispiel & Beispiel\\
  Beispiel & Beispiel & Beispiel & Beispiel & Beispiel & Beispiel & Beispiel\\
  Beispiel & Beispiel & Beispiel & Beispiel & Beispiel & Beispiel & Beispiel\\
  Beispiel & Beispiel & Beispiel & Beispiel & Beispiel & Beispiel & Beispiel\\
  Beispiel & Beispiel & Beispiel & Beispiel & Beispiel & Beispiel & Beispiel\\
  Beispiel & Beispiel & Beispiel & Beispiel & Beispiel & Beispiel & Beispiel\\
  Beispiel & Beispiel & Beispiel & Beispiel & Beispiel & Beispiel & Beispiel\\
  Beispiel & Beispiel & Beispiel & Beispiel & Beispiel & Beispiel & Beispiel\\
  Beispiel & Beispiel & Beispiel & Beispiel & Beispiel & Beispiel & Beispiel\\
  Beispiel & Beispiel & Beispiel & Beispiel & Beispiel & Beispiel & Beispiel\\
  Beispiel & Beispiel & Beispiel & Beispiel & Beispiel & Beispiel & Beispiel\\
  Beispiel & Beispiel & Beispiel & Beispiel & Beispiel & Beispiel & Beispiel\\
  Beispiel & Beispiel & Beispiel & Beispiel & Beispiel & Beispiel & Beispiel\\
\end{MRTtable}

\subsection{Blabla}
\subsection{Blök}
\subsection{Blubb}

\section{Programmlistings}
Für die Einbindung von Code würde ich das Paket \textrm{listings} mit der
\env{lstlisting} empfehlen und nicht die Verwendung der \env{verbatim}, wie hier
angewandt.
% slightly reduced spacing (looks better here, IMHO)
\bgroup\MRTthesisSetup{stretch text=1.2}
%\begin{verbatim}>>>
\begin{verbatim}
from mpl_pgfpre import *
from random import seed,random
from draw_map import draw_map
seed()

fig    = draw_map()
cloud1 = ((-1.8,3.8),(1,3))
cloud2 = ((0.8,3.9),(4.2,7))
cloud3 = ((-1.8,0.3),(3.6,7.2))
cloud  = (cloud1,cloud2,cloud3)
points = ([],[],[])
for i in range(3):
    for j in range(20):
        points[i].append(\
            [random()*(cloud[i][0][1]-cloud[i][0][0])+cloud[i][0][0],\
             random()*(cloud[i][1][1]-cloud[i][1][0])+cloud[i][1][0]])
        plt.plot(points[i][j][0],points[i][j][1],"ko")

savefig("img/2b")
\end{verbatim}%<<<
\egroup

\section{Bilder}
\begin{minipage}{\textwidth}
  \centering
  % example-image-duck should be part of modern LaTeX distributions. It is
  % contained in the latest TeXLive and MikTeX versions
  \includegraphics
    [width=.9\textwidth, height=.8\textwidth, keepaspectratio=true]
    {example-image-duck}
  \captionof{figure}{Großes Bild einer Ente im Anhang}
\end{minipage}
\endgroup
