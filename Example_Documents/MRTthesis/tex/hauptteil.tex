\begingroup
\color{blue}
\chapter{Hauptteil}
Bilder werden auch automatisch durchnummeriert, wenn sie in einer \env{figure}
eingebunden und mit einer \cs{caption} versehen werden. Mit \verb|\ref{label}|
oder \verb|\autoref{label}| können sie referenziert werden. Beispiel:
\verb|\autoref{fig:box}| produziert „\autoref{fig:box}“.
\begin{figure}[htbp]
  \centering
  \fbox{\phantom{\rule{4cm}{2cm}}}
  \caption[Testbild: Eine rechteckige Box]
    {%
      Testbild: Eine rechteckige Box. Der restliche Text dieser Überschrift wird
      nicht im Bildverzeichnis erscheinen, da das optionale Argument von
      \cs{caption} verwendet wurde.\label{fig:box}
    }
\end{figure}

Das Ganze funktioniert auch bei Tabellen wie bei \autoref{tab:test}, die mit der
\env{MRTtable} gesetzt wurde.

\begin{MRTtable}
  [
    ,cap=
      {%
        Testtabelle mit der \env{MRTtable}. Der weitere Text wird nicht im
        Tabellenverzeichnis erscheinen, da der Key \texttt{scap} verwendet
        wurde.%
      }
    ,scap={Testtabelle mit der \env{MRTtable}}
    ,label=tab:test
    ,col=lcl
    ,pos=htbp
  ]
  Begriff & Symbol & Erläuterung\\
  Murgs & $\mu$ & Weiß ich selber nicht\\
  Abmurgs & $a$ & ???
\end{MRTtable}
\endgroup
