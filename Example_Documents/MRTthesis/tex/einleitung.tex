\chapter{Einleitung}
\begingroup
\color{blue}
Absatzformate werden nicht benötigt, \LaTeX\ erkennt automatisch, ob es sich um
den ersten Absatz nach einer Überschrift handelt oder nicht, und rückt
dementsprechend die erste Zeile ein oder nicht. Während des Tippens braucht man
sich um die Formatierung von normalem Text wenig Gedanken machen.

Dieser Absatz ist zum Beispiel eingerückt. Dinge, um die man sich Gedanken
machen sollte, sind vor allem Sonderzeichen, die in \LaTeX\ bzw.\@ \TeX\ für
Befehle vorgesehen sind, wie bspw.\@ '\%', '\textbackslash', '\_' oder
'\^{}'. Diese lassen sich durch Befehle wie: \verb|\%|, \verb|\textbackslash|,
\verb|\_| oder \verb|\^{}| setzen. Eine kleine Übersicht liefert folgender Link%
\footnote{\url{https://de.wikibooks.org/wiki/LaTeX-Kompendium:_Sonderzeichen}}.

\noindent
Mit \verb|\noindent| lässt sich die Einrückung eines Absatzes gezielt
verhindern.

Mit \verb|begin{enumerate}...\end{enumerate}| lassen sich nummerierte
Aufzählungen setzen.
\begin{enumerate}
  \item Mit \verb|itemize| anstelle von \verb|enumerate| kann man Strichpunkte
    machen
    \begin{itemize}
      \item hier in einer verschachtelten Form
      \item Man kann die Art der Aufzählung auch beeinflussen:
        \begin{enumerate}[label=\alph*)]
          \item Beispielsweise so
            \begin{enumerate}[label=(\roman*)]
              \item oder so
            \end{enumerate}
        \end{enumerate}
      \item[*•>] willkürliches Angeben des Symbols funktioniert auch
    \end{itemize}
  \item Die Möglichkeiten sind nahezu unbegrenzt
\end{enumerate}

\section{Stand der Technik}
Das ist ein üblicher Gliederungspunkt.

\section{Motivation und Zielsetzung}
Das auch.

\color{cyan!50!blue}
\section{Das ist eine extrem hässliche und lange, ja gerade zu in epischer
Breite ab\-ge\-fasste Über\-schrift, die wahrscheinlich umgebrochen werden
muss.}
Nur zur Verdeutlichung, wie mit sehr langen Überschriften umgegangen wird.
Absichtlich in einem schmutzigen Cyan gesetzt, um der Hässlichkeit Ausdruck zu
verleihen. Leider funktioniert der Algorithmus zur automatischen Silbentrennung
von \LaTeX\ in unterstrichenen Überschriften nicht, weshalb die Silbentrennung
hier manuell mit \verb|\-| im Wort angegeben werden muss.

Es ist wichtig eine Kurzform anzugeben, damit die Kopfzeile weiterhin schön
gesetzt und nicht mehrzeilig wird. Eine Kurzform ist durch
\verb|\section[short]{long}| anzugeben. Man beachte, dass durch dieses Vorgehen
auch die Kurzform ins Inhaltsverzeichnis übernommen wird. Ist dies nicht
erwünscht, kann

\verb|\section{long}\markright{\thesection\ short}|

\noindent
verwendet werden. Für Kapitel sollte allerdings

\verb|\chapter{long}\markboth{\thechapter\ short}{\thechapter\ short}|

\noindent
oder 

\verb|\chapter{long}\sethead{\thechapter\ short}|

\noindent
gebraucht werden. Selbiges gilt sinngemäß auch für andere Gliederungsebenen. Man
beachte, dass die Argumente von \verb|\sethead|, \verb|\markboth|,
\verb|\markleft| und \verb|\markright| nicht darauf überprüft werden, ob es
sich tatsächlich um Gliederungsebenen handelt. Nahezu jeglicher Inhalt ist hier
theoretisch möglich..
\endgroup
